
\subsubsection{Experiments and their research methodology}


In the context of quantitative empirical research, the starting point for experiments is a preliminary theoretical work that results in the formulation of hypotheses. These hypotheses are often found in the form that certain conditions or sets of conditions are put forward as influencing factors for a phenomenon and causal assumptions are made (\cite{Gniewosz.2011}). Experimental research and field research are often based on these causal assumptions. However, the procedure for testing these assumptions is different. In field research, phenomena (dependent variable) are supposed to be predicted by measuring influencing factors (independent variable). Independent variables or influencing factors could therefore be for example \enquote{good teaching} whereas the resulting phenomena or dependent variable would be \enquote{good grades} for the students. This posterior approach raises problems in deriving causal conclusions in many correlational studies, because effects of third variables can never be completely excluded. To address this problem, in experimental designs, the expression of the independent variable is not measured but generated by experimental manipulations. This means that the expression of the independent variable is triggered by a specific experimental design.
The influence of independent variables on the dependent variable, which can be clearly demonstrated without confounding factors, is referred to as internal validity. Laboratory experiments, which ideally eliminate all external factors, therefore usually have a very high internal validity. However, due to the fact that laboratory experiments can only represent a small part of the total population in an artificially created environment, they usually have a very low external validity. This circumstance is usually exactly the opposite for field research or field experiments. During a field experiment, not all external factors can be eliminated, which is why only a low internal validity can be guaranteed. Nevertheless, field experiments usually better represent the overall population, which is why they have a higher external validity. Another key feature of experiments is the elimination of confounding variables. Confounding variables are influencing variables that can also affect the dependent variable and thus affect the relationship between the independent and dependent variable. This influence must be neutralized or controlled for the planned experiment in order to identify the effects of the independent variable (\cite{Gniewosz.2011}). There are two types of confounding variables, those that depend on the individual participants of the experiment and those that arise from the experimental design itself (\cite{Gniewosz.2011}). In order to neutralized these confounding variables a number of different approaches exist, which are listed in table \ref{tab:confounding_variables}.

\begin{table}[htbp]
    \centering
    \begin{tabular}{|L{0.49\textwidth}|L{0.49\textwidth}|}
    \cline{1-2}
    Confounding variable                                          & Description \\ \cline{1-2}
    Parallelization of confounding variables of the test subjects & The confounding variables of all test subjects are measured beforehand. the participants are then divided among the experimental conditions in such a way that the mean values of the confounding variable are approximately equal in all experimental conditions.         \\ \cline{1-2}
    Randomization of confounding variables of the subjects        & If it is not known which confounding variables might be relevant, the test subjects can be randomly assigned to experimental conditions, which from a probabilistic point of view, results in an equal distribution over the experimental conditions.     \\ \cline{1-2}
    Removing confounding variables of the experimental situation      & If it is known which experimental conditions triggers the confounding variables, these conditions should be eliminated in the experimental situation            \\ \cline{1-2}
    Keeping constant confounding variables of the experimental situation  & All confounding variables for all test subjects should have the same form         \\ \cline{1-2}
    Random variation of confounding variables of the experimental situation & If the confounding variables cannot be removed or kept constant they can be randomly assigned over the whole experimental process. The confounding variable would vary randomly between the experimental conditions and thus not be able to mask the effect of the manipulated independent variable.     \\ \cline{1-2}
    Control group                    &  Another option to rule out confounding variables is the usage of a control groups           \\ \cline{1-2}
    Control of expectation effects                   &  Not telling participants and personnel performing the experiment the final goal of the experiment can counteract expectations, which could otherwise influence the outcome as confounding variables           \\ \cline{1-2}
    \end{tabular}
    \caption[Counter measures for confounding variables]{Counter measures for confounding variables in experimental research (\cite{Gniewosz.2011})}
    \label{tab:confounding_variables}
    \end{table}

If these confounding variables are fully removed an experiment can be conducted where the test subjects assignment to the experimental conditions is completely determined. If these confounding variables are not fully removed the effects of other influences cannot be completely ruled out. Experiments under these conditions are refered to as quasi-experiments.  
Furthermore, there are other characteristics by which experiments are categorized. For instance Single Factorial and Multifactorial Designs describe experiments that test a single or multiple dependent variables respectively. If the independent variable is subject ot the experiment on the other hand, the experiment categories as an univariate or multivariate a design. On top of that the distinction between \enquote{between subject} and \enquote{within subject} designs are made. \enquote{Within subject} designs describe experiments where the test subjects are tested multiple times with different experimental setups, whereas \enquote{between subject} experiments test different subjects with different experimental setups (\cite{Gniewosz.2011}). 