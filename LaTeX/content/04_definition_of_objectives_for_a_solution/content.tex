\newpage\section{Definition of Objectives for a solution}\label{sec:objectForSolution}

\subsection{Literature Review Studies in Data Analytics and General}

\subsubsection{Requirements Elicitation}

\subsubsection{Requirements Specification}

\subsubsection{Requirements Validation}


Personen müssen aufgeklärt werden über das Experiment \cite{Dresch.2011}


\subsection{Existing Resources for Online Experiments}

%The findings from the analysis of other applications and measures that could have optimized the research process in the field of data analytics are fundamentally important for the creation of the artifact of this thesis. Therefore, after showing that there is a gap in the literature in the area of data analytics and especially the carrying out of experiments, in this part several already existing applications are presented which could have been used for carrying out experiments. The identified problem and the found literature gap are only further emphasized, since the question arises, if there have already been applications that are supposed to support the execution of experiments, why have they not been used in the field of data analytics. None of the literature analyzed above used any of the following applications. The following applications were taken from a list of for Resources for Online Experiments at Columbia University (\cite{Columbia.2023}).

\subsubsection{z-Tree}
Zurich Toolbox for Readymade Economic Experiments

\subsubsection{oTree}
An open-source platform for laboratory, online, and field experiments

\subsubsection{LIONESS Lab}
a free web-based platform for conducting interactive experiments online

