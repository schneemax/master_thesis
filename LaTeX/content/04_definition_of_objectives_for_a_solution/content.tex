\newpage\section{Definition of Objectives for a solution}\label{sec:objectForSolution}

As described earlier, goals for solving the problem are described by requirements, which are established using the \enquote{requirement engineering} approach. Beginning with the \textit{Requirements Elicitation} phase, applications and frameworks are reviewed, which enable the conducting of experiments. Subsequently, a literature review is conducted to establish further requirements for the field of data analytics based on previous research. Information gathered in this step are then used in the \textit{Requirements Specification} step in order to specify requirements for the final artefact. These requirements are then validated in the \textit{Requirements Validation} phase.

\subsection{Requirements Elicitation}

This phase gathers information in order to discorver possible requirements for the final artefact.

\subsubsection{Existing Resources for Online Experiments}

In this section, just three different resources for conducting online experiments are presented and then analyzed. A small example experiment was implemented with each of the applications in order to find out the advantages and disadvantages of the respective resource. The goal of this section is to draw conclusions and define requirements from the applications that allow to design an artifact for the field of data analytics. The applications analyzed are resources recommended for experiments by the Columbia Experimental Laboratory for Social Sciences. The selection of resources analyzed in the course of this thesis is limited to applications that allow the implementation of interactive experiments. Resources or applications that are only suitable for conducting surveys or passive experiments were neglected.

\textbf{z-Tree}
Zurich Toolbox for Readymade Economic Experiments

alt läuft nur auf windows computern (Nachteile bei oTree schauen)
nicht open source
kaum grafik interface support

\textbf{oTree}
An open-source platform for laboratory, online, and field experiments

wird eher für Field Experiments eingesetzt (interne und externe validität?) open source
benötigt server, client gerät und coding

\textbf{LIONESS Lab}
a free web-based platform for conducting interactive experiments online

kaum coding benötigt? --> schlecht, custimzing ist nicht besser
Roboter feature (simuliert participant input) <-- cooles feature
konzentriert sich ebenfalls mehr auf field experiments
nicht open source
platform  hängt komplett an den herausgebern




\subsubsection{Studies in Data Analytics - A Literature Review }



\subsection{Requirements Specification}

\subsection{Requirements Validation}


Personen müssen aufgeklärt werden über das Experiment \cite{Dresch.2011}




%The findings from the analysis of other applications and measures that could have optimized the research process in the field of data analytics are fundamentally important for the creation of the artifact of this thesis. Therefore, after showing that there is a gap in the literature in the area of data analytics and especially the carrying out of experiments, in this part several already existing applications are presented which could have been used for carrying out experiments. The identified problem and the found literature gap are only further emphasized, since the question arises, if there have already been applications that are supposed to support the execution of experiments, why have they not been used in the field of data analytics. None of the literature analyzed above used any of the following applications. The following applications were taken from a list of for Resources for Online Experiments at Columbia University (\cite{Columbia.2023}).




