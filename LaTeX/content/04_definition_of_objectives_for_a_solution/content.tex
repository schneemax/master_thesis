\newpage\section{Definition of Objectives for a solution}\label{sec:objectForSolution}

\subsection{Literature Review Studies in Data Analytics and General}

\subsection{requirements elicitation}

\subsubsection{Functional and non-functional requirements}

Damit eine Anwendungsmodernisierung durchgeführt und eine modernisierte Architektur konzeptioniert werden kann, ist die Bestimmung der Anforderungen wichtig.\autocite[Vgl.][Kapitel 3]{Seacord.2003} Anforderungen können nach ISO/IEC 25000, beziehungsweise dem Qualitätsmodell aus ISO/IEC 25010, als Qualitätskriterien an Software und Systeme klassifiziert werden.\autocite[Vgl.][]{ISOIEC25010.2011} Das \ac{ieee} definiert Anforderungen als eine Bedingung oder Eigenschaft, welche von einem System oder einer Systemkomponente erfüllt werden muss, um eine Problemstellung oder Zielsetzung eines Nutzers oder formalen Dokuments zu erfüllen.\autocite[Vgl.][S.62]{IEEE.1990} Anforderungen können nach diesen beiden Definitionen als zu erfüllende Eigenschaften oder Qualitätskriterien einer Software oder eines Systems definiert werden. Aus diesem Grund werden die Anforderungen an den SAP Lean Catalog aus den von der SAP beschriebenen Produkteigenschaften und Merkmalen des SAP Lean Catalogs abgeleitet. Zusätzlich werden allgemeine Anforderungen an eine Softwaremodernisierung berücksichtigt, welche erfüllt sein müssen, um eine erfolgreiche Migration durchzuführen. Diese Anforderungen an den SAP Lean Catalog und an eine Softwaremigration im Allgemeinen werden weiter in funktionale und nichtfunktionale Anforderungen eingeteilt. %\info{Was sind Anforderungen? Welche werden eroben?}

Eine funktionale Anforderung beschreibt eine Funktion oder Fähigkeit eines Systemes, die konkret von einem System oder einer Softwarekomponente durchgeführt werden können muss.\autocite[Vgl.][S.35]{IEEE.1990} Ein Beispiel für eine funktionale Anforderung wäre die Berechnung des Bestellpreises in Euro und in Dollar. Nichtfunktionale Eigenschaften beschreiben hingegen Verhaltensweisen des Systems\autocite[Vgl.][Kapitel 3]{Seacord.2003} und gehen damit über die funktionalen Eigenschaften hinaus. Damit beschreiben funktionale Anforderungen was ein System können muss und nichtfunktionale Anforderungen wie es funktionieren soll. Nichtfunktionale Anforderungen beschreiben außerdem häufig die Qualität der Funktionen und können mehrere andere Anforderungen beeinflussen.\autocite[Vgl.][S.109ff]{Balzert.2011} Ein Beispiel für nichtfunktionale Eigenschaften wäre, dass die Umrechnung von Euro in Dollar in \enquote{wenigen Sekunden} durchgeführt werden muss.

%\subsubsection{Non-functional Requirements}

\subsection{Requirements analysis}


Personen müssen aufgeklärt werden über das Experiment \cite{Dresch.2011}

