\newpage\section{Identification of the Problem}\label{sec:identification_of_the_problem}

This part of the thesis defines the problem to be solved by the \ac{dsr} approach. For this purpose, a literature search is first conducted to find potential gaps in the field of data analytics and methods used in this field. Subsequently, applications and tools are examined, which had been intended to support the research process.

In order to identify constrains on the research on data analytics a literature search is conducted. The main objective of it is to analyze the existing literature to find research gaps, particularities and interrelationships between literature. This is supposed to give insights into the current state of research and to find out which part of the research process on data analyitcs still has room for improvements. The literature search itself is conducted in the field of boundaries and conflicts that might hinder the usage of data analytics. As already presented, data analytics has become an extremely important topic for companies. Therefore, the identification and resolution of obstacles that limit the use of data analytics is fundamentally important for companies. For this reason and to expand the scope of the literature search, in addition to reviewing relevant literature on data analytics, an exploration of literature dealing with boundaries in data analytics is pursued. Consequently, relevant literature was identified and reviewed. Afterwards, the identified literature was categorized and analyzed. Initially, it was assumed that the topic of data analytics lies both in the field of information systems and business (\cite{Abbasi.2016}, \cite{Levina.2005}). For this reason, the literature search was mainly conducted in literature databases that focussed on these topics. Table \ref{literature_search_db} shows the databases that were used. 

The literature search was conducted using a keyword search. The used keywords consist of phrases like \enquote{Data Analytics}, \enquote{Data} and \enquote{Boundary}. A full list of keywords that were used is included in the appendix. %In addition, the search was limited to articles only. 


In order to ensure the quality of the identified literature initially, only publications from certain journals were considered. These journals consist of the \textit{Senior Scholars' Basket of Journals} and the \textit{UT Dallas Top 100 Business School Research Rankings}. The former includes journals in the area of information systems and the later includes journals in the area of business administration. A full list of included journals is listed in the appendix. Furthermore, only peer-reviewed articles were taken into account. This was done to ensure the quality of the found publications and to additionally exclude book reviews, editorials and opinion statements. Moreover, other 'non-scholarly' texts or publications that did not meet scientific requirements were also not considered in the search. Secondly, the abstracts of the particular articles were inspected to narrow the search further. Consequently, literature that did not meet the topic of boundaries in data analytics was excluded from the search. %After initially only considering articles that were published in one of the mentioned journals, the search was extended to other publications as long as these publications also met scientific requirements and were officially published. 
The literature found in the search was then used for a backward and forward search. During a backward search, all cited sources of an article are examined and during a forward search all the literature that cites the original article is examined (\cite{Webster.2002}). The backward search was conducted using Google Scholar. In addition to this, articles from other journals were, in a second step, reviewed and included as well if they met the scientific requirements, were officially published and relevant to the topic. This process yielded 35 research publications. The results were then assigned to different phases of the aforementioned information value chain, their content best represents. This was done to find literature gaps in the general process of data processing. Additionally, the identified literature was categorized by their research methodology in order to find patterns and similarities in the literature. 
As stated before, the information value chain consists of the phases \enquote{data}, \enquote{information}, \enquote{knowledge}, \enquote{decisions} and \enquote{actions}. The found literature was assigned to these phases, in order to structure and analyze the findings. %The intention behind this being that the information value chain represents the process of data processing according to the current state of knowledge. 
By mapping the literature found, parts of the data processing process that are over- or under-represented may become visible. From this, conclusions can be drawn about the current state of research. Furthermore, the categories \enquote{overspanning} and \enquote{other} were introduced to represent literature that either fits multiple phases of the information value chain or none. Using this method leads to the results shown in the \enquote{First Search} column of table \ref{informationValueChainResults}.

\begin{table}[htbp]
    \centering
    \small
    \begin{tabular}{lcc}
    \hline
    \multicolumn{1}{c}{Information Value Chain}  & First Search & \multicolumn{1}{l}{Additional Search} \\ \hline
    Data                                         & 4            &                                       \\
    Information                                  & 3            &                                       \\
    Knowledge                                    & 21           &                                       \\
    Decisions                                    & 4            & 0                                      \\
    Actions                                      & 0            & 0                                      \\
    Overspanning                                 & 0            & 3                                     \\
    Other                                        & 3            &                                       \\ \hline
    \textbf{Total}                               & 35           & 3                                     \\ \hline
    \end{tabular}
    \caption{Results Assigned to the Information Value Chain}
    \label{informationValueChainResults}
    \end{table}

Table \ref{informationValueChainResults} shows an overabundance of literature that got assigned to the \enquote{knowledge} phase of the information value chain. %Among other things, this is due to the fact that the content of this literature deals with the construction and exchange of knowledge within certain groups. %The context of this literature is mostly not directly written within the context of data analytics, but nonetheless deals with boundaries in a relevant context. %Literature concerning this subject exists in abundance.
The significantly fewer entries for the other phases indicate less research on these sub-parts of the information value chain. However, it cannot be concluded that this underrepresentation is due to the fact that these phases are less relevant in the context of data analytics. For this, more literature would have to exist confirming that these areas are less important for data analyitics or less prone to boundaries. 
The underrepresentation of the phases \enquote{data} and \enquote{information} could also be explained by the fact that these phases are more technology driven and therefore less researched in a bigger organizational data analytics context. In fact, the corresponding literature, which was assigned to these phases mainly consists of publications researching the technical possibilitties and application of data. Their main research object does not directly consist of any broader topics for companies or the application of Data Anayltics. The focus of this literature is largely to overcome technical hurdles or to show how individual technical functions can be implemented.
This circumstance is particularly worrying in the context of data analytics, since Amankwah-Amoah and Adomako (\cite{AmankwahAmoah.2019}) has already established that the mere existence of data does not yet have any added value for the organization. Research in these very technical areas, such as how data is generated in the first place, is therefore fundamentally important, but without further research it does not contribute significantly to the effective use of data analytics.
Nonetheless, in total, seven individual publications could be found that fit into these two phases. In addition, these two phases (\enquote{data} and \enquote{information}) are mostly considered together in the further elaboration, since the literature which was assigned to these phases lies thematically very closely together. 

Only four publications were assigned to the \enquote{decisions} phase and none to the \enquote{actions} phase. These results in particular call into question if the topic of behavioral research in data analytics has been extensively researched. The reason for this is the fact that data analytics is primarily a decision support method (\cite{Runkler.2020}). Therefore, an overabundance of literature delineating the decision-making process of data analytics should likely exist. This is compounded by the fact that no literature could be found that addressed overspanning issues, as no overarching theories could exist for an insufficiently studied topic. In order to ensure that the ratio of the literature found is based on the research state and not on the keyword search being biased in any way, a second literature search was conducted focussed on finding more literature that could be assigned to the \enquote{decisions} or \enquote{actions} phase. This was only done for these phases as these two are most relevant in the context of data analytics and because, in total, the least literature could be assigned to them (viewing \enquote{data} and  \enquote{information} together). This second keyword search was conducted with the goal of finding more literature that could be assigned to the phases \enquote{decisions} and \enquote{actions}. Therefore, a new set of keywords including \enquote{decision}, \enquote{decision making} and \enquote{action} were added to the existing set of keywords. The full list of keywords is included in the appendix. Furthermore, the abstracts were examined with an emphasis on the aforementioned goal. The results of this second keyword search are represented in the \enquote{Additional Search} column of table \ref{informationValueChainResults}. A total number of three additional publications were identified using this second search. These three publications were all assigned to the \enquote{overspanning} category. Consequently, no additional literature that could be assigned to the phases \enquote{decisions} or \enquote{actions} could be identified. This further indicates the fact that the topic of boundaries in data analytics is not researched extensively. 

A total number of 38 publications were identified in these two searches and analyzed further.

%\subsection{Research Type and Methods}

In order to further analyze the literature and to potentially draw further conclusions, the found literature was also categorized regarding the research method that was used. This categorization is presented in table \ref{researchMethod}. 

\begin{table}[htbp]
    \centering
    \small
    \begin{tabular}{llc}
    \hline
    \multicolumn{1}{l}{Research approach} & \multicolumn{1}{l}{Method} & \multicolumn{1}{l}{Number} \\ \hline
    Qualitative (22)                      & Case Study                 & 13                         \\
                                          & Interviews                 & 4                          \\
                                          & Experiments                & 2                          \\
                                          & Observation                & 3                          \\
    Quantitative (16)                     & Survey                     & 12                         \\
                                          & Data Analysis              & 6                          \\ \hline
    \end{tabular}
    \caption{Research Approach Used in the Literature}
    \label{researchMethod}
    \end{table}

The distribution presented in table \ref{researchMethod} show significant discrepencies in the number of research methods used. Methods such as case studys and surveys are used more frequently than average in contrast to other methods. 
The least frequently used methodology is the experiment. This fact further indicates an insufficient exploration of the field of behavioral research in data analytics, as experiments are the most suitable method for researching the behavior of people (\cite{Gniewosz.2011}). In this context experiments are a particularly important tool for investigating causal relationships in research (\cite{Gniewosz.2011}).


As already indicated in section \ref{subsec:informationValueChainSubSection}, much of the literature found could be assigned to the \enquote{knowledge} phase of the information value chain. %This literature mostly consists of publications which research boundaries in certain professional groups, but not within the context of data analytics. 
In addition, literature can be identified which was assigned to the information value chain phases \enquote{data} and \enquote{information}. This literature mostly consists of technical publications, whose main goal is to research areas of application and advantages of data analytics. These publications, are for the most part, only concert with very specific technical issues. These two research areas, which do not quite fit the context of data analytics, reinforce the assumption that this field is not researched extensively. Other indications that have already been mentioned are the lack of literature that overspans the topic as a whole and the lack of literature that can be assigned to the \enquote{decisions} and \enquote{actions} phases of the information value chain. This circumstance is supported by the fact that hardly any experiments could be found as a used method in the current literature, even though experiments should be an integral part of behavioral research in the field of data analytics (\cite{Gniewosz.2011}). This has not only created a blind spot in the current state of knowledge, but also a permanent limitation for new research, which lacks existing knowledge as a foundation for new science. %The results   and parties between which the boundaries occur, seem to confirm this.
In summary, the literature search indicates a gap in the literature on behavioral research on data analytics as well as a lack of experimental research in this area. Although current knowledge suggests that both behavioral and experimental research should be an integral part of data analysis research, the literature review of this section was able to show that both areas are extremely underrepresented in literature and research. This circumstance is further underlined by the fact that resources already exist to support experimental research in the field of behavioral science (\cite{Columbia.2023}), but none of the articles analyzed utilized them. 

In this section, the artifact is designed and developed based on the previously created requirements. At this point, the specific form of the artifact should also be discussed. Basically, no software component has to be developed based on the Design Science Research approach. A theoretical concept or a methodology can also arise from the Design Science Research approach. However, the conceptualized artifact should correspond to a form that addresses the original problem. As has already been shown, the original problem is the lack of experimental research in the field of data analytics. Section \ref{sec:objectForSolution} shows that there is a need for support for the experimental setup in existing applications as well as in already conducted studies. The literature search in section \ref{sec:identification_of_the_problem} as well as the literature search in section \ref{sec:objectForSolution} does not point to conceptual problems with the methodology in these studies. Supporting the execution of experiments therefore seems to be best implemented by an IT artifact instead of a methological concept or other possibilities for an artifact. For this reason, requirements for the direct experimental setup are recorded and these are then used to implement the artifact as a software component, with the goal of improving the performance of experiments.

