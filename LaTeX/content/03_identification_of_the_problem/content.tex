\newpage\section{Identification of the Problem}\label{sec:identification_of_the_problem}

\subsection{Previous Studies and Gaps in the Literature}

%\subsubsection{Literature Search: Salient boundaries and conflicts in data analytics}
%\subsubsection{Salient boundaries and conflicts in data analytics}
%Literature Review: Behavioral Research in Data Analytics (on the basis of the Information Value Chain)

%This, however, might be hindered by boundaries and conflicts that have arisen during an organization's existence. Technical challenges are only some part of the underlying problem. Boundaries that have surfaced between employees or professional groups might also impact the effective usage of data analytics. This problem only becomes more apparent when the fact is taken into account that most companies are not built around processing data, but have subsequently implemented some sort of data analytics system into their organization and processes.

%The effective use of data and utilization of data analytic tools might be the next digital revolution, which businesses have to adapt to, in order to stay competitive. This literature review focusses on the salient boundaries and conflicts that potentially restrain companies from utilizing their data to its fullest potential. In an effort to achieve this, the current state of knowledge regarding boundaries in data analytics is analysed. For this, a literature review was conducted. The goal of this literature review, above all else, is to summarize the current knowledge of the topic, to find interrelationships that span the literature and to show potential literature gaps for future research on the topic.

%In order to analyze and structure the existing literature effectively, this elaboration firstly defines the terms \enquote{data analytics}, \enquote{boundaries} and \enquote{conflicts}. In addition, the so-called \enquote{information value chain} is introduced, which will be used to structure the found literature further. Afterwards, the research method and procedure of the literature review are presented. Eventually, the findings and results are discussed and a conclusion about the current state of research on the topic is given.

%\newpage\subsubsubsection{Research Method}

In order to identify constrains on the research on data analytics a literature search is conducted. The main objective of it is to analyze the existing literature to find research gaps, particularities and interrelationships between literature. This is supposed to give insights into the current state of research and to find out which part of the research process on data analyitcs still has room for improvements.

%to get an indication of the current state of research and to find out where in the research process there is room for improvement.

The literature search itself is conducted in the field of boundaries and conflicts that might hinder the usage of data analytics.Consequently, relevant literature was identified and reviewed. Afterwards, the identified literature was categorized and analyzed.Initially, it was assumed that the topic of boundaries and conflicts in data analytics lies both in the field of information systems and business (\cite{Abbasi.2016}, \cite{Levina.2005}). For this reason, the literature search was mainly conducted in literature databases that focussed on these topics. Table \ref{literature_search_db} shows the databases that were used. 

%to find previous studies and identify gaps in the literature.    a Previous Studies and Gaps in the Literature

%analyse the current state of knowledge regarding the topic of data analytics in general a literature was conducted. The main objective was to analyze the existing literature to find research gaps, particularities and interrelationships between literature.


%boundaries and conflicts in data analytics, a literature search was conducted. The main objective was to analyze the existing literature to find research gaps, particularities and interrelationships between literature. Consequently, relevant literature was identified and reviewed. Afterwards, the identified literature was categorized and analyzed. 



%Since, the topic of this literature review lies both in the field of informatics and in the field of business, the literature search was mainly conducted in literature databases that focus on business, informatics or business informatics. Table \ref{literature_search_db} shows the databases that were used.      %, leading to the identification of possible research gaps or particularities.

\begin{table}[htbp]
    \centering
    \begin{tabular}{lL{8cm}}
    \hline
    Online database           & Subject Focus                                                          \\ \hline
    ABI/INFORM Collection     & Business and management                                                \\
    Business Source Premier   & Accounting, business, economics, management                            \\
    EconBiz                   & Business and economics                                                 \\
    ProQuest One Business     & Business                                                               \\
    AIS Electronic Library    & Informatics                                                            \\
    MIS Quarterly Website     & Business informatics                                                   \\
    Web of Science            & Multiple databases that provide access to different academic topics    \\
    Google Scholar            & Web search engine for scholarly literature across an array of disciplines \\ \hline
    \end{tabular}
    \caption{Databases Used in the Literature Search}
    \label{literature_search_db}
    \end{table}


The literature search was conducted using a keyword search. The used keywords consist of phrases like \enquote{Data Analytics}, \enquote{Data} and \enquote{Boundary}. A full list of keywords that were used is included in the appendix. %In addition, the search was limited to articles only. 


In order to ensure the quality of the identified literature initially, only publications from certain journals were considered. These journals consist of the \textit{Senior Scholars' Basket of Journals} and the \textit{UT Dallas Top 100 Business School Research Rankings}. The former includes journals in the area of information systems and the later includes journals in the area of business administration. A full list of included journals is listed in the appendix. Furthermore, only peer-reviewed articles were taken into account. This was done to ensure the quality of the found publications and to additionally exclude book reviews, editorials and opinion statements. Moreover, other 'non-scholarly' texts or publications that did not meet scientific requirements were also not considered in the search. Secondly, the abstracts of the particular articles were inspected to narrow the search further. Consequently, literature that did not meet the topic of boundaries in data analytics was excluded from the search. %After initially only considering articles that were published in one of the mentioned journals, the search was extended to other publications as long as these publications also met scientific requirements and were officially published. 
The literature found in the search was then used for a backward and forward search. During a backward search, all cited sources of an article are examined and during a forward search all the literature that cites the original article is examined (\cite{Webster.2002}). The backward search was conducted using Google Scholar. In addition to this, articles from other journals were, in a second step, reviewed and included as well if they met the scientific requirements, were officially published and relevant to the topic. This process yielded 35 research publications. The results were then assigned to different phases of the aforementioned information value chain, their content best represents. This was done to find literature gaps in the general process of data processing. Additionally, the identified literature was categorized by their research methodology and by their respective industry and involved departments in order to find patterns and similarities in the literature. For example, the overaverage occurrence of boundaries in certain industries could indicate that certain businesses are more susceptible to the emergence of boundaries. In a second in-depth analysis, the different boundaries and possible solutions that were proposed by these articles were categorized and summarized in order to draw overarching conclusions.


%The number of initial results of the database search were extremly high, likely because of the broad nature of the keywords. An example for this would be the keyword \enquote{data}, which resulted in a high number of unrelevant findings. For this reason all database query results were manually examined. 

%Using this process a total number of 35 articles were identified and assigned to the different phases of the information value chain. The results were then categorized by their research methology, results, and by their respective industry and involved departments. 


%With the purpose of structuring the literature further and identifying possible research gaps, the literature was also assigned to different steps of the information value chain. This 



%\newpage\subsubsection{Findings}

The findings of the literature search were structured in multiple ways in order to draw further conclusions than the individual publications allow for. Therefore, the literature search was structured according to the information value chain, the methodology used and the industry covered. Subsequently, the commonalities in content of the literature and the conclusion of the discussed articles are presented.

%\subsection{Information Value Chain}\label{informationValueChainSubSection}

As stated before, the information value chain consists of the phases \enquote{data}, \enquote{information}, \enquote{knowledge}, \enquote{decisions} and \enquote{actions}. The found literature was assigned to these phases, in order to structure and analyze the findings. %The intention behind this being that the information value chain represents the process of data processing according to the current state of knowledge. 
By mapping the literature found, parts of the data processing process that are over- or under-represented may become visible. From this, conclusions can be drawn about the current state of research. Furthermore, the categories \enquote{overspanning} and \enquote{other} were introduced to represent literature that either fits multiple phases of the information value chain or none. Using this method leads to the results shown in the \enquote{First Search} column of table \ref{informationValueChainResults}.

\begin{table}[htbp]
    \centering
    \begin{tabular}{lcc}
    \hline
    \multicolumn{1}{c}{Information Value Chain}  & First Search & \multicolumn{1}{l}{Additional Search} \\ \hline
    Data                                         & 4            &                                       \\
    Information                                  & 3            &                                       \\
    Knowledge                                    & 21           &                                       \\
    Decisions                                    & 4            & 0                                      \\
    Actions                                      & 0            & 0                                      \\
    Overspanning                                 & 0            & 3                                     \\
    Other                                        & 3            &                                       \\ \hline
    \textbf{Total}                               & 35           & 3                                     \\ \hline
    \end{tabular}
    \caption{Results Assigned to the Information Value Chain}
    \label{informationValueChainResults}
    \end{table}

Table \ref{informationValueChainResults} shows an overabundance of literature that got assigned to the \enquote{knowledge} phase of the information value chain. Among other things, this is due to the fact that the content of this literature deals with the construction and exchange of knowledge within certain groups. The context of this literature is mostly not directly written within the context of data analytics, but nonetheless deals with boundaries in a relevant context. %Literature concerning this subject exists in abundance.

The significantly fewer entries for the other phases could be explained due to these phases being researched less. However, it cannot be concluded that this underrepresentation is due to the fact that these phases are less susceptible to boundaries. For this, more literature would have to exist confirming that these areas are less prone to boundaries. The underrepresentation of the phases \enquote{data} and \enquote{information} could also be explained by the fact that these phases are more technology driven and therefore less researched in the context of boundaries. In fact, the corresponding literature, which was assigned to these phases mainly consists of publications researching the application of big data. Their main research object does not directly consist of the identification or resolution of boundaries. Nonetheless, in total, seven individual publications could be found that fit into these two phases. In addition, these two phases (\enquote{data} and \enquote{information}) are mostly considered together in the further elaboration, since the literature which was assigned to these phases lies thematically very closely together.

Only four publications were assigned to the \enquote{decisions} phase and none to the \enquote{actions} phase. These results in particular call into question if the topic of boundaries in data analytics has been extensively researched. The reason for this is the fact that data analytics is primarily a decision support method (\cite{Runkler.2020}). Therefore, an overabundance of literature delineating the decision-making process of data analytics should likely exist. This is compounded by the fact that no literature could be found that addressed overspanning issues, as no overarching theories could exist for an insufficiently studied topic. In order to ensure that the ratio of the literature found is based on the research state and not on the keyword search being biased in any way, a second literature search was conducted focussed on finding more literature that could be assigned to the \enquote{decisions} or \enquote{actions} phase. This was only done for these phases as these two are most relevant in the context of data analytics and because, in total, the least literature could be assigned to them (viewing \enquote{data} and  \enquote{information} together). This second keyword search was conducted with the goal of finding more literature that could be assigned to the phases \enquote{decisions} and \enquote{actions}. Therefore, a new set of keywords including \enquote{decision}, \enquote{decision making} and \enquote{action} were added to the existing set of keywords. The full list of keywords is included in the appendix. Furthermore, the abstracts were examined with an emphasis on the aforementioned goal. The results of this second keyword search are represented in the \enquote{Additional Search} column of table \ref{informationValueChainResults}. A total number of three additional publications were identified using this second search. These three publications were all assigned to the \enquote{overspanning} category. Consequently, no additional literature that could be assigned to the phases \enquote{decisions} or \enquote{actions} could be identified. This further indicates the fact that the topic of boundaries in data analytics is not researched extensively. 

A total number of 38 publications were identified in these two searches and analyzed further.

%\subsection{Research Type and Methods}

In order to further analyze the literature and to potentially draw further conclusions, the found literature was also categorized regarding the research method that was used. This categorization is presented in table \ref{researchMethod}. 

\begin{table}[htbp]
    \centering
    \begin{tabular}{llc}
    \hline
    \multicolumn{1}{l}{Research approach} & \multicolumn{1}{l}{Method} & \multicolumn{1}{l}{Number} \\ \hline
    Qualitative (22)                      & Case Study                 & 13                         \\
                                          & Interviews                 & 4                          \\
                                          & Experiments                & 2                          \\
                                          & Observation                & 3                          \\
    Quantitative (16)                     & Survey                     & 12                         \\
                                          & Data Analysis              & 6                          \\ \hline
    \end{tabular}
    \caption{Research Approach Used in the Literature}
    \label{researchMethod}
    \end{table}

The distribution presented in table \ref{researchMethod} does not show any significant results. Methods such as case studys and surveys are used more often, this however might be due to the fact that these research methods are easier to implement or more common. Experiments for example might be harder to justify and less effective than surveys, in the context of data analytics. %This ration might also be explained through the ration of literature that was found. %The distribution of literature found favoured the \enquote{knowledge} phase of the information value chain. Assuming that case studies and surveys are used in this type of literature, this would explain the ratio of research methods used. 

%\subsection{Affected Industry and Department}

%The articles were also classified according to their respective industries and departments they discuss. This was done to potentially identify patterns in the occurrence of bonudaries.

%\begin{sidewaystable}[]
%    \centering
%    \begin{tabular}{l|cccccccccc|c}
%        \multicolumn{1}{c|}{}                                                                                 & \multicolumn{10}{c|}{Industry}                                                                                                                                                                                                                                                                                              & \multicolumn{1}{l}{\textbf{Total}} \\ \cline{2-12} 
%        \multicolumn{1}{c|}{\multirow{2}{*}{\begin{tabular}[c]{@{}c@{}}Occurrence of \\ boundaries\end{tabular}}} & \multicolumn{1}{c|}{\begin{tabular}[c]{@{}c@{}}Primary \\ Sector\end{tabular}} & \multicolumn{2}{c|}{\begin{tabular}[c]{@{}c@{}}Secundary \\ Sector\end{tabular}} & \multicolumn{6}{c|}{\begin{tabular}[c]{@{}c@{}}Tertiary \\ Sector\end{tabular}}              & \multicolumn{1}{l|}{}                                    & \multicolumn{1}{l}{}               \\ \cline{2-12} 
%        \multicolumn{1}{c|}{}                                                                                 & \multicolumn{1}{c|}{\begin{tabular}[c]{@{}c@{}}Oil and\\  Gas\end{tabular}}    & Production                    & \multicolumn{1}{c|}{Software}                    & Financial & Transport & Insurance & Consulting & Education & \multicolumn{1}{c|}{Healthcare} & \begin{tabular}[c]{@{}c@{}}Not \\ specified\end{tabular} &                                    \\ \hline
%        \begin{tabular}[c]{@{}l@{}}Upper\\ Management\end{tabular}                                            & \multicolumn{1}{c|}{}                                                          & 2                             & \multicolumn{1}{c|}{}                            &           & 1         &           &            &           & \multicolumn{1}{c|}{}           & 4                                                        & 7                                  \\ \cline{1-1}
%        \begin{tabular}[c]{@{}l@{}}Different \\ Organizations\end{tabular}                                    & \multicolumn{1}{c|}{}                                                          &                               & \multicolumn{1}{c|}{}                            & 1         &           &           &            &           & \multicolumn{1}{c|}{1}          & 1                                                        & 3                                  \\ \cline{1-1}
%        \begin{tabular}[c]{@{}l@{}}Knowledge \\ Sharing\end{tabular}                                          & \multicolumn{1}{c|}{1}                                                         &                               & \multicolumn{1}{c|}{1}                           & 2         &           & 1         & 1          &           & \multicolumn{1}{c|}{1}          &                                                          & 7                                  \\ \cline{1-1}
%        \begin{tabular}[c]{@{}l@{}}Different \\ Departments\end{tabular}                                      & \multicolumn{1}{c|}{}                                                          & 2                             & \multicolumn{1}{c|}{}                            &           &           &           &            &           & \multicolumn{1}{c|}{2}          & 2                                                        & 6                                  \\ \cline{1-1}
%        \begin{tabular}[c]{@{}l@{}}Information \\ Systems\end{tabular}                                        & \multicolumn{1}{c|}{}                                                          & 1                             & \multicolumn{1}{c|}{1}                           &           & 1         &           &            &           & \multicolumn{1}{c|}{}           & 2                                                        & 5                                  \\ \cline{1-1}
%        \begin{tabular}[c]{@{}l@{}}Local \\ Teams\end{tabular}                                                & \multicolumn{1}{c|}{}                                                          & 2                             & \multicolumn{1}{c|}{1}                           & 1         &           & 1         & 1          & 1         & \multicolumn{1}{c|}{1}          & 2                                                        & 10                                 \\ \hline
%        \textbf{Total}                                                                                        & \multicolumn{1}{c|}{1}                                                         & 7                             & \multicolumn{1}{c|}{3}                           & 4         & 2         & 2         & 2          & 1         & \multicolumn{1}{c|}{5}          & 11                                                       & 38                                 \\ \hline
%        \end{tabular}
%    \caption{Industry and Subject Matter of the Literature}
%    \label{industryTable}
%    \end{sidewaystable}

%Table \ref{industryTable} shows the assignment of the identified literature to the corresponding area that was the topic of the particular publication. Specifically, the literature was divided by areas or objects in which or by which boundaries occurred and in which industry this was studied. This was done to assess whether boundaries do occur above average in certain areas or industries. The distribution of the location boundaries are occurring in, shown in table \ref{industryTable} (column \enquote{Occurrence of boundaries}), is relatively balanced. Boundaries occurring between different organizations is slightly less often topic of publications and boundaries between local teams slightly more often. However, this circumstance is likely due to the fact that studies spanning multiple organizations are comparatively more difficult to conduct and that local teams were often the topic of literature that deals with boundaries in knowledge management, which was more common in the results of the literature search. In fact, most boundaries seem to occur between groups that are part of a clearly separate affiliation. For example, upper management can be prone to lead to boundaries as this group of people is separated from the workforce by organizational structures. Simultaneously, local teams within the workforce can also be separated by organizational structures and thus be prone to boundaries. The rows \enquote{Knowledge Sharing} and \enquote{Information Systems} that are displayed in table \ref{industryTable} show that this exceeds the occurrence of boundaries between groups. Results from the literature indicate that the detachment of processes and systems from the natural workflow or groups in general also encourages boundaries. %promote if incentiveses doesnt work


%also confirm this as boundaries are also prone to occur when systems or processes are detached from each other or from groups. An example of this would be process to input data to later use for data analyitics applications, which is not part of the usual workflow of an employee and can thus lead to boundaries.

%refer to literature that researched the occurrence of boundaries within or between separated processes (knowledge management and information systems) and other parts of the organizaiton. 

%The distribution of literature regarding the corresponding industry is less evenly divided. Industries in the primary sector are the most underrepresented in this context, only being the focus of one publication in the oil and gas industry. The most common industry, on the other hand, is the traditional production industry, which includes all businesses that are engaged in the production or manufacturing of goods. A reason for this could be the widespread availability of these traditional industries. More patterns like the distribution of capital in different industries were tried to be applied to the results displayed in table \ref{industryTable}, but none resulted in any significant results. 

%For example the distribution of literature to various industries does not represent any profit allocation or ratio to other industries. The distribution of literature most likely originates from the fact how accesable certain kinds of industries are. 
%In summary, no further particularities can be drawn from the distribution of literature among the corresponding industries.


%\newpage
%\subsection{Textual Findings}

%After analyzing the literature in terms of its methodology and field of application, the specific results of each publication were analyzed in-depth. Consequently, in the following, both commonalities in boundaries and solutions to solve them are presented. These are structured according to common features and show the number of publications that support these findings. Table \ref{boundariesInIndustry} shows the most common boundaries that are researched in literature and table \ref{resolveBoundary} is the most common means to resolve them.

%In general, all results regarding the occurrence of boundaries can be divided into the three categories indicated in table \ref{boundariesInIndustry}. These three categories all have in common that some kind of separation takes place between groups. Another finding that has emerged from the literature is that the occurrence of boundaries between separate parties does not only affect groups of people. Separating people and processes through poor tasks or separating people and IT systems through poor processes can also lead to boundaries. The literature indicates that any form of unclean separation between parties inevitably leads to boundaries. This is confirmed by literature which shows that boundaries occur even when processes or systems whose task it is to minimize boundaries are too isolated. On the other hand, tightly integrated processes show little or no boundaries. The literature gives the supply chain as an example of this. Although a large number of different groups and departments have to work together in it, there are (usually) hardly any major boundaries, since the individual parties are closely connected by the optimized affiliation to the supply chain (\cite{Chen.2021}).


%\begin{table}[htbp]
%    \centering
%    \begin{tabular}{L{0.238\linewidth}L{0.58\linewidth}c}
%    \hline
%    Theoretical construct                                      & Meaning                                                                                                                                                                                                                                                                                                                & \multicolumn{1}{l}{Articles} \\ \hline
%    Between local groups and professional teams                & Boundaries often occur between groups that are part of a clearly separated affiliation, like management and workforce, offshore teams, different organizations or even employees who are supposed to interact as boundary spanners (\cite{Levina.2008}, \cite{vanOsch.2016}, \cite{Makela.2019}).                                                                 & 16 \\
%    Information Technology and available data                  & The IT systems or data do not meet the quality required for effective use (\cite{Wixom.2001}). Additionally, in some cases employees are not able to use the full potential of the systems due to lack of training (\cite{Goodhue.1992}).                                                                                                                         & 10 \\ 
%    Too much investment in dedicated boundary spanning systems & The investment into dedicated systems or actions that are meant to span boundaries can in of itself lead to boundaries (\cite{Levina.2005}). A critical condition for this to occur is the isolated un-integrated implementation of such systems. Examples found in the literature range from IT infrastructure (\cite{Currie.2004}) to boundary spanning staff). & 3  \\ \hline                                                           
%    \end{tabular}
%    \caption{Boundaries Found in the Literature}
%    \label{boundariesInIndustry}
%\end{table}

%\begin{table}[htbp]
%    \centering
%    \begin{tabular}{L{0.238\linewidth}L{0.58\linewidth}c}
%    \hline
%    Theoretical construct                                      & Meaning                                                                                                                                                                                                                                                                                                                & \multicolumn{1}{l}{Articles}                                                                                                                                                              \\ \hline
%    Integration of personnel and processes                     & The integration of people and information technology can effectively prevent boundaries from forming (\cite{Kotlarsky.2014}). Information systems should be naturally used in different tasks, instead of requiring the additional use of a dedicated system (\cite{Levina.2005}, \cite{Wook.2021}). The close collaboration between teams and organizations can prevent boundaries at the same time (\cite{Zhang.2021}).                                                                                      & 14 \\
%    Intrinsic motivation of individuals and teams              & Boundary spanning employees need to be encouraged in order to make them more effective. This can be done through organizational design of processes or tasks or reward and feedback mechanisms (\cite{Minbaeva.2018}). The boundary spanners should aim for the larger mutual benefit of organizations (\cite{Makela.2019}). One example is that semi-formal boundary spanners are often more effective than formal ones, because of their intrinsic motivation to share knowledge (\cite{GuvenUslu.2020}).   & 10 \\
%    Right usage of big data analytics                          & In order to unlock the full potential of (big) data analytics it is important to utilize it right (\cite{AmankwahAmoah.2019}). This can be, for example, archieved through the usage and data scientists (\cite{Kim.2021}) or the usage of the right application architecture (\cite{Goodhue.1992}).                                                                                                                                                                                                            & 8  \\
%    Quality of data                                            & Deployed information system must allow for a seamless data collection (\cite{Lukyanenko.2019}). Especially the data quality in the area of Big Data must be guaranteed (\cite{Wook.2021}).                                                                                                                                                                                                                                                                                                                    & 5  \\
%    Usage of boundary objects                                  & The usage of boundary objects, such as information technology or documentation of all kinds can improve boundary spanning capabilities (\cite{Pawlowski.2004}).                                                                                                                                                                                                                                                                                                                                               & 5  \\ 
%    Legitimacy of boundary spanning personnel                  & The fact that boundary spanners are formally nominated and socially accepted is an important part of their effectiveness (\cite{Levina.2005}). Based on this, it is important that employees know who can be addressed about certain topics (\cite{Mell.2022}).                                                                                                                                                                                                                                                   & 2  \\ \hline
%    \end{tabular}
%    \caption{Solutions for Boundaries Found in the Literature}
%    \label{resolveBoundary}
%\end{table}

%Table \ref{resolveBoundary} shows possible solutions to overcome boundaries that are suggested by the literature. These are more varied and numerous, but can also be summarized. The most important means of counteracting  boundaries is the close collaboration between different groups, processes, and systems, as well as intrinsic motivation of stakeholders and quality of data and the right usage of systems. A large part of literature which directly deals with the occurrence of boundaries also mentions the implementation of so called boundary spanners and boundary objects as a way to reduce boundaries. Cross and Parker define boundary spanners as \enquote{[...] critical links between two groups of people that are defined by functional affiliation, physical location, or hierarchical level [...]} (\cite{Cross.2004}). %page 74
%Thus, boundary spanners represent individuals that unite any two groups that are separate from each other. This concept is, for example, used in knowledge management to explain the distribution and cooperation of knowledge sharing in organizations (\cite{Levina.2005}). Managers can be one example of these kinds of individuals, as they are supposed to bridge the gaps between multiple teams through a hierarchical manner (\cite{Allen.1969}). IT employees can be another more informal example of boundary-spanning employees as they have to interact with a number of different teams which maybe would not share a connection otherwise. In this case, the IT-system itself can act as a so-called boundary-object (\cite{Pawlowski.2004}). Boundary-Objects are any kind of object that is used by a variety of distinct groups. An example of a boundary object would be the sketch of a car, which is created by the design team and shared with an engineering team to calculate wind resistance. The main property of boundary-objects that makes them help resolve boundaries is that they are used by different groups which otherwise do not share any extensive and direct interactions with each other. The presence of both boundary spanners and boundary-objects in literature indicates that close collaboration and cooperation can lead to the resolving of boundaries. This is supported by the other findings of table \ref{resolveBoundary}.

%\newpage

%\subsection{Discussion}
%\newpage\subsubsection{Gaps in the literature}

As already indicated in section \ref{informationValueChainSubSection}, much of the literature found could be assigned to the \enquote{knowledge} phase of the information value chain. This literature mostly consists of publications which research boundaries in certain professional groups, but not within the context of data analytics. In addition, literature can be identified which was assigned to the information value chain phases \enquote{data} and \enquote{information}. This literature mostly consists of technical publications, whose main goal is to research areas of application and advantages of data analytics. These publications, are for the most part, not concerned with potential boundaries that could arise within data analytics. These two research areas, which do not quite fit the context of boundaries in data analytics, reinforce the assumption that this field is not researched extensively. Other indications that have already been mentioned are the lack of literature that overspans the topic as a whole and the lack of literature that can be assigned to the \enquote{decisions} and \enquote{actions} phases of the information value chain. %The results   and parties between which the boundaries occur, seem to confirm this.
%The methodologies and research approaches as well as the departments in which the research was conducted in do seem to confirm this. 

%In addition these people and processes must be intrinsically motivated for this tight integration. 
%Although the exact subject of this literature review is not researched extensively, as the literature search showed, general boundaries and solutions for them could be found (table \ref{resolveBoundary} and \ref{boundariesInIndustry}). These findings show 

%Nevertheless, it was possible to find boundaries and solutions for them in the literature. As already mentioned, this literature is not written in the context of data analyitics, which again indicates that this topic is less researched. 

%Nevertheless, the literature that can be identified regarding boundaries in organizations shows that a lot of boundaries occur between groups that are part of a clearly separated affiliation. As already analyzed, this circumstance applies not only to groups of people but also to processes and systems. An interesting conclusion from this would be to no longer consider boundaries as an event that occurs, but as a default state that can be contained by certain measures. This would lead to the assumption that boundaries exist between each employee or asset in a company and that these can be mitigated through good processes, tight integration or IT systems, instead of having the assumption that boundaries only occur after the fact. This would shift the research focus from identifying reasons for boundaries to studying how to actively solve them, which might be more effective.

%The remaining literature that could be identified deals less with boundaries and more with data analytics itself. In particular, big data analytics is a rather new topic and thus a lot of research focusses on its potential rather than on what kind of boundaries or conflicts hold it back. In addition, most data analytic studies start out with a set of data and then use it to explore the possibilities of big data, without regard to the whole process. This general approach might not be suited to show boundaries, because the used data already exists. An approach where a certain desired outcome is specified first and then data is collected, preferably in an organizational context, might be better suited. This approach, however, was not implemented in any of the literature, possibly due to the fact that research is still trying to determine what capabilities the usage of big data and data analytics has. This also reflects the aforementioned fact that no literature could be found that could have been assigned to the \enquote{decisions} and \enquote{actions} phases of the information value chain. An interesting approach could therefore be to research the occurrence of boundaries in the decision-making process. Under certain circumstances, it would be possible that decisions based on data analytics are trusted differently than the ones made by more traditional means. The predictions of a machine learning algorithm are maybe more trusted than the ones by experienced managers. This might lead to some internal conflicts and would therefore be an example of a currently insufficiently studied part of the topic of boundaries in data analytics. 

%Furthermore, it should be again critically noted that although a large number of boundaries and solutions for them could be found in the literature, almost all of these were not examined in the context of data analytics. The results are nevertheless transferable to the topic of data analytics, although slightly different actual results in the area of data analytics are conceivable and will only become apparent with further research into this area.

%All of this leads to the fact that boundaries in a traditional organizational context are researched quite extensively. However, boundaries in the context of data analytics are not. In general, although findings can be transferred thematically, the exact facts still contain many gaps in the literature. 

%As mentioned before, the three goals of this literature review were to summarize the current knowledge of the topic of boundaries in data analytics, to find interrelationships that span the literature and to show potential literature gaps for future research on the topic.

%\newpage\section{Conclusion}
%Even though data analytics might be the next big thing and shape the future of decision making in business, the topic in of itself is only researched fragmentarily. The literature of data analytics mainly focusses on its application. Possible boundaries that limit the effectiveness of data analytics are not discussed. Generally, the occurrence of boundaries in organizations and possible counter measures to resolve them are presented in literature. These, however, focus mostly on knowledge sharing in a traditional setting and do not treat the subject matter in the context of data analytics. These two different literature paths narrow down the topic of boundaries in data analytics, but do not explore it directly. This shows a gap in the current state of research that could be subject to future investigation. Simultaneously, even though literature could be found discussing boundaries and challenges in data analytics, almost no literature could be found that focusses on the decision-making process itself. This is suprising as data analytics is first and, for most, a decision-making tool. This circumstance again clearly shows the lack of research done in this area and another research gap. Future investigations could, for example, focus on the effects of data analytics on the decision-makers themselves. Nevertheless, it was possible to find literature whose findings can be transferred thematically to the field of data analytics. This literature clearly indicates that boundaries tend to occur between groups that do not share an affiliation. Simultaneously, it was found that information technology or other processes can generally lead to boundaries if they are not integrated well enough. These boundaries can, however, be overcome through the tight integration of the different parties. Finally, intrinsic motivation and the quality of processes and data are further factors that can influence the occurrence of boundaries and conflicts. These general findings reflect and summarize the current state of knowledge regarding boundaries in data analytics.

%In summary, it was found that the topic has not yet been extensively researched. At the same time, commonalities in terms of boundaries and solutions were found in the literature and transferred to the topic of data analytics. This is highly relevant, because data analytics is an important topic for companies to remain competitive. For this reason, the area of boundaries in data analytics should be further investigated.

%The three main goals of this literature review, summarizing the current state of knowledge, finding interrelationships that span literature and to show literature gaps could therefore be met. In particular, identifying gaps in the literature for future research is an important milestone in order to fully understand how salient boundaries and conflicts in data analytics can be resolved.

\subsection{Applications (Anwendungen) for Behavioral Research}