\newpage\section{Introduction}

\subsection{Background and Motivation}

%The introduction and widespread usage of new technologies that utilize the collecting of huge data sources like data analytics or machine learning has already been profen to be disruptive for all industries. The amount of data generated globally is rising at the same time (\cite{Seagate.2018}) and the pressure to use these data volumes effectively in order to gain a business advantage rises. This new trend, often coined \enquote{big data} after the fact that never before seen amounts of data are generated and are available for processing, enables completely new business areas. This is reinforced, among other things, by the fact that many companies already view their data as a primary business asset (\cite{Redman.2008}). Simultaneously, the emergence of big data promises to completely reshape the decision-making process of traditional businesses through the adoption of data analytics. Although sales in the area of big data have risen significantly over the past years (\cite{BISResearch.2018}, \cite{Bitkom.2018}) and businesses already view big data as an important information technology trend (\cite{Bitkom.2017}) a lot of organizations struggle to effectivly utilize their data. Some 84\% of industry-leading companies in the United States and around the world were already investing in big data analytics in 2019, according to their own statements, which only underlines the importance of data analytics for decision-making in the economy. (\cite{statista.2019}). This is also reinforced by the market for big data analytics worlwide expected to more then double in size in the next 6 years (\cite{statista.2022}).

The introduction and widespread adoption of new technologies that utilize the collection of huge data sources, like data analytics or machine learning, has already been proven to be disruptive for all industries. The amount of data generated globally is increasing rapidly (\cite{Seagate.2018}), and the pressure to effectively utilize these data volumes to gain a competitive advantage is rising. This new trend, often referred to as \enquote{big data}, after the fact that never-before-seen amounts of data are generated and available for processing, enables completely new business areas. This is reinforced, among other things, by the fact that many companies already view their data as a primary business asset (\cite{Redman.2008}). Simultaneously, the emergence of big data promises to completely reshape the decision-making process of traditional businesses through the adoption of data analytics. Although sales in the area of big data have risen significantly over the past years (\cite{BISResearch.2018}, \cite{Bitkom.2018}), and businesses already view big data as an important information technology trend (\cite{Bitkom.2017}), a lot of organizations struggle to effectively utilize their data. Some 84\% of industry-leading companies in the United States and around the world were already investing in big data analytics in 2019, according to their own statements, which only underlines the importance of data analytics for decision-making in the economy (\cite{statista.2019}). This is also reinforced by the market for big data analytics worldwide expected to more than double in size in the next 6 years (\cite{statista.2022}).


%In their article, Amankwah-Amoah and Adomako study the influence of big data usage on business failure. They come to the conclusion that the mere possession of big data as an asset has no positive effects on an organization (\cite{AmankwahAmoah.2019}) and that in order to prevent business failure, big data must be used effectively (\cite{AmankwahAmoah.2019}). Conducting research to resolve the underlying factors hindering the efficient utilization of data analytics in this particular context holds therefore crucial significance. However, although decision-making in data analytics has recently attracted scholars' attention (\cite{Chen.2022}) there is still a lack of research on non-technical aspects holding back the utilization of data analytics. Section \ref{sec:identification_of_the_problem} of this thesis uses the information value chain to look at the state of research in data analytics. Specifically, a literature review in the field of Data Analytics is conducted. The results of this literature review show that there is a lack of research on non-technical aspects of the information value chain. These literature gaps mainly consist of areas like decision-making and behavioral research. These results have been confirmed by studies in the past focussing on related fields (\cite{Trieu.2017}), which might indicate a persistant issues. This only becomes more apparent as new technologies, which overlap with data analytics like machine learning and artificial intelligence get more widespread. These \enquote{black-box} technologies are already alternatives to human decision-making (\cite{Krakowski.2023}). Moreover other studies confirm that aspects like company culture, business models and the overall commitment and strategy of organizations have a big impact on the effectiveness of data analyitics (\cite{Holsapple.2014}). This lack of research on non technical regards could become a huge issue in the future, specifically for the decision-making process as firms' top priorities focus more and more on big data analytics for strategic decision making (\cite{Ghasemaghaei.2019}). Furthermore, it is indicated in section \ref{sec:identification_of_the_problem} that the current state of research specifically lacks a variety of experiments conducted to confirm the validity of frameworks and hypothesis. The experiment being a particularly important tool for investigating causal relationships in research (\cite{Gniewosz.2011}). In addition other means of collecting information like inquest questionnaires, which are probably the most frequently used form of obtaining information (\cite{Mummendey.2014}) in quantitative research are not always the most suitable method. The behavior of people, for example, which comprises the literature gap found, can be better assessed by means of observational studies or experiments (\cite{Gniewosz.2011}). While prior research has conducted behavioral research of data analytics with surveys and case studies, little or no attention has been paid to the verification of research results and hypothesis through experiments. This general lack of experimental research in certain areas connects the applied business problem of better utilizing data analytics in organizations to the theoretical problem of lack of experimental research generally found in data analytics (refering to the results of the literature search in section \ref{sec:identification_of_the_problem}). Improving the experimental research process in the field of data analytics could thereby significantly improve the future state of knowledge on said topic, whilste also allowing organizations to succeed in the fast paste economical environment of the digital-age.

In their article, Amankwah-Amoah and Adomako study the influence of big data usage on business failure. They conclude that the mere possession of big data as an asset has no positive effect on an organization (\cite{AmankwahAmoah.2019}) and that, in order to prevent business failure, big data must be used effectively (\cite{AmankwahAmoah.2019}). Conducting research to resolve the underlying factors hindering the efficient utilization of data analytics in this particular context holds crucial significance. However, although decision-making in data analytics has recently attracted scholars' attention (\cite{Chen.2022}), there is still a lack of research on non-technical aspects hindering the utilization of data analytics.

Section \ref{sec:identification_of_the_problem} of this thesis uses the information value chain to examine the state of research in data analytics. Specifically, a literature review in the field of data analytics is conducted. The results of this literature review reveal a lack of research on non-technical aspects of the information value chain, mainly in areas like decision-making and behavioral research. These results have been confirmed by studies in the past focusing on related fields (\cite{Trieu.2017}), which might indicate persistent issues. This becomes more apparent as new technologies, which overlap with data analytics like \ac{ml} and \ac{ai}, become more widespread. These \enquote{black-box}\footnote{Technologies whose exact internal structure is unknown.} technologies being already an alternative to human decision-making (\cite{Krakowski.2023}). Moreover, other studies confirm that aspects like company culture, business models, and the overall commitment and strategy of organizations have a significant impact on the effectiveness of data analytics (\cite{Holsapple.2014}).

This lack of research on non-technical aspects could become a significant issue in the future, specifically for the decision-making process, as firms' top priorities increasingly focus on big data analytics for strategic decision-making (\cite{Ghasemaghaei.2019}). Furthermore, it is indicated in Section \ref{sec:identification_of_the_problem} that the current state of research specifically lacks a variety of experiments conducted to confirm the validity of frameworks and hypotheses. Experiments are a particularly important tool for investigating causal relationships in research (\cite{Gniewosz.2011}). In addition, other means of collecting information, like inquest questionnaires, which are probably the most frequently used form of obtaining information (\cite{Mummendey.2014}) in quantitative research, are not always the most suitable method. The behavior of people, for example, which comprises the literature gap found, can be better assessed by means of observational studies or experiments (\cite{Gniewosz.2011}). While prior research has conducted behavioral research of data analytics with surveys and case studies, little or no attention has been paid to the verification of research results and hypotheses through experiments. This general lack of experimental research in certain areas connects the applied business problem of better utilizing data analytics in organizations to the theoretical problem of a lack of experimental research generally found in data analytics (referring to the results of the literature search in Section \ref{sec:identification_of_the_problem}). Improving the experimental research process in the field of data analytics could significantly enhance the future state of knowledge on this topic, while also allowing organizations to succeed in the fast-paced economic environment of the digital age.


%\footnote{Technologies whose exact internal sequence can hardly or not at all be explained, which therefore are acting like a \enquote{black-box}}

\subsection{Objective and Expected Contribution}

%The objective of this thesis is therefore to improve the experimental research process in the field of behavioral research in data analytics. This is done through the following three objectives: (1) the review of prior research on data analytics and their methodological procedure (2) the development of an artefact which improves the research process in the field of behavioral research in data analytics (3) the validation of the artefact through the exemplary realization of a study in said field utilizing said artefact. In order to accomplishe the creation of this artefact the \acf{dsr} methodology is used, which contains the six steps, \textit{Identification of the Problem}, \textit{Definition of Objectives for a solution}, \textit{Design and Dev of artefacts}, \textit{Demonstration of the Artifact}, \textit{Evaluation of the solution} and \textit{Communication} (\cite{Peffers.2006}, \cite{Dresch.2015}). Based on these steps, the corresponding artifact is conceptualized. For this purpose, a literature review in the field of data analytics is first conducted to identify research gaps and underlying problems in the area of data analytics. Subsequently, a further literature review and other sources are used to establish requirements that the conceptualized artifact must meet in order to solve these research gaps and underlying problems. This is accomplished by identifying literature in the field of data analytics that uses experiments or whose research object would in principle have permitted the use of experiments. These insights are then used to conceptualize, analyze and validate requirements for the final artefact, utilizing the \textit{Requirement Engineering} approach (\cite{Sommerville.2011}, \cite{SWEBOK.2004}). These requirements are then used to design and develop an artefact. Subsequently, the resulting artefact is then demonstrated and assessed by implementing a real experimental study in data analytics as an example to evaluate the artefact and its benefits for the experimental research process. This examplary implementation is then also used to validate the afformentioned requirements. The last step of the \ac{dsr} framework, which focusses on communicating the results to its stakeholders, is ensured by this thesis itself. The practical contribution of this thesis to research is twofold. On the one hand an artefact is created which accelerates research in the field of data analytics, through the improvement of the research process. On the other hand meta-knowledge about the research process itself is created, which not only improves the conduct of research through said artefact, but can also be used in off-topic areas beyond the use cases of this thesis.

The objective of this thesis is, therefore, to improve the experimental research process in the field of behavioral research in data analytics. This is done through the following three objectives: (1) the review of prior research on data analytics and its methodological procedure, (2) the development of an artifact that improves the research process in the field of behavioral research in data analytics, and (3) the validation of the artifact through the exemplary realization of a study in said field utilizing said artifact. In order to accomplish the creation of this artifact, the \acf{dsr} methodology is used, which contains the six steps: \textit{Identification of the Problem}, \textit{Definition of Objectives for a Solution}, \textit{Design and Development of Artifacts}, \textit{Demonstration of the Artifact}, \textit{Evaluation of the Solution}, and \textit{Communication} (\cite{Peffers.2006}, \cite{Dresch.2015}). Based on these steps, the corresponding artifact is conceptualized. For this purpose, a literature review in the field of data analytics is first conducted to identify research gaps and underlying problems in the area of data analytics. Subsequently, a further literature review and other sources are used to establish requirements that the conceptualized artifact must meet to solve these research gaps and underlying problems. This is accomplished by identifying literature in the field of data analytics that uses experiments or whose research object would, in principle, have permitted the use of experiments. These insights are then used to conceptualize, analyze, and validate requirements for the final artifact, utilizing the \textit{Requirement Engineering} approach (\cite{Sommerville.2011}, \cite{SWEBOK.2004}). These requirements are then used to design and develop an artifact. Subsequently, the resulting artifact is demonstrated and assessed by implementing a real experimental study in data analytics as an example to evaluate the artifact and its benefits for the experimental research process. This exemplary implementation is also used to validate the aforementioned requirements. The last step of the \ac{dsr} framework, which focuses on communicating the results to its stakeholders, is ensured by this thesis itself. The practical contribution of this thesis to research is twofold. On the one hand, an artifact is created that accelerates research in the field of data analytics through the improvement of the research process. On the other hand, meta-knowledge about the research process itself is created, which not only improves the conduct of research through said artifact but can also be used in off-topic areas beyond the use cases of this thesis.
