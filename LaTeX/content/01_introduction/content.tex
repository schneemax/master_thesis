\newpage\section{Introduction}

\subsection{Background and Motivation}

Over the past few years, data analytics has become increasingly important for companies across all industries. With the massive amount of data that is now available, companies can use data analytics to gain valuable insights into consumer behavior, market trends, and internal operations, among other things. As a result, data analytics has become a critical tool for companies looking to gain a competitive edge in today's rapidly evolving business environment. However, while data analytics has become an essential tool for businesses, there has been relatively little research done in the area of behavioral research. 

In the past couple of years this led many companies and researchers into spending an above-average share of resources into technical areas like data extraction, preparation or harmonization. While this has laid a very good foundation for the topic of data analytics both in businesses and in research, it has also led to topics relating to the use of data beyond technical aspects being neglected. 

Specifically, there is a lack of research on the decision-making process involved in data analytics, and how individuals and organizations use data analytics to inform their decisions.




Looking at how companies utilize their available data an above-average share of resources is dedicatded to actions like data extraction or preparation, whilste areas beyond these more technical processes hardly get any attention. In the past couple of years this led to big investments 


%\subsection{Research Problem and Objectives}

\subsection{Objective and methodology of the work}

 One of the major challenges in conducting research in this area is the high cost of developing custom applications for each study. The development of such applications can be time-consuming, expensive, and often requires specialized expertise. To address this challenge, this thesis develops a generic application that streamlines the process of conducting studies in the field of data analytics.

\subsection{Contribution and Scope of the Study}

This application enables researchers to design, conduct, and analyze studies more efficiently and cost-effectively, allowing them to explore the field in greater depth. This will be accomplished by using the design science research approach. Firstly, the problem of a lack of behavioral research in data analytics is identified. Then, the objectives for a solution are defined through a literature review and the use of requirement engineering to gather requirements for the application. Next, the application is design, implemented prototypically and its functionality demonstrated. Finally, solution is evaluated through the usages of the requirements.

%The role of information systems in \ac{it} modern business solutions is indisputable, [...] Developing such a solution is the goal of %this work \parencite{venkatesh_usability_2014}. 

%Process mining needs to access [...] and used by many large-scale companies \parencite{hoehle_espoused_2015} across the world.

%And here we demonstrate like \parencite{hoehle_espoused_2015} how citations look alike in this \LaTeX file. You can also list all authors \parencite{venkatesh_usability_2014}. And click any of the references and see what happens in your PDF reader, like here: \textcite{university_of_arkansas_mobile_2015}.