\newpage\section{Conclusion}

This section will first summarize the findings of the study, followed by an overview about the contribution and implications of the thesis for the current state of research, and finally a recommendation for future research in the area of experimental research in data analytics is given. 

\subsection{Summary of the Study}

The goal of this thesis was the development of an artifact which would improve Experimental Research in Behaviroal research in Data Analytics. For this purpose the \ac{dsr} approach was used. First, the current state of knowledge in the field of data analytics was searched with the help of a literature review. The focus was primarily on literature that deals with obstacles in the use of data analytics in the context of organizations. The results of this literature search were then assigned to the different phases of the information value chain. This assignment reveals a research gap in the field of behavioral research in the context of data analytics. At the same time, a research gap in the experimental setups of the literature could be found. This indicates a below average perpetuation of experiments. The two problems identified, a lack of behavioral research in the field of data analytics and a lack of experiments as a method in studies, were then used to find targets for a solution to these problems. In a new literature search, all literature in the field of data analytics was identified that either used experiments or allowed the use of experiments in principle through their methodology. In this way, literature was identified that provides information about possible problems and challenges in the use of experiments in studies in the field of data analytics. Furthermore, we looked at existing solutions that can be used to conduct experiments and other sources that provided information about experiments in general. These sources were then used to establish requirements. These were first identified from the literature found and then processed using the findings from the analysis of existing applications and other sources. The requirements for the artifact have thus reflected a precise list of requirements for an application with the help of which experiments in the field of behavioral research in data analytics can be carried out. Test cases for the individual requirements were also set up for the validation of the requirements. These can be used to verify the requirements themselves and at the same time serve to validate the final artifact. Subsequently, the artifact was conceptualized. For this purpose, processes for the artifact were first conceptualized in a general and technology-independent manner. These processes are based on the basic functionalities that where discovered by the requirements. Subsequently, Android Studio in combination with Java was selected as the technologies for implementing the artifact. The requirements were again the decisive factor in the selection of the technologies. In the next step, a technical architecture for the artifact was developed. Best practices as well as prerequisites defined by the requirements were taken into account. The fully conceptualized architecture was then implemented and deployed. The implementation of an example study also demostrated the functions and features of this conceptualized artifact. The implemented artifact was then validated. For this purpose, the individual test cases were checked on the one hand, and the requirements that were set up were examined in detail on the other. In addition, the user interface elements were demonstrated by means of a practical test in which 31 participants navigated through the user interface. No errors or deficiencies in the design of the user interface were found. Overall, the artifact could be validated for all requirements in this way.

\subsection{Contributions and Implications}

The artifact designed in this thesis represents a framework that can be used to conduct experiments in the field of behavioral research in the area of data analytics.

\subsection{Future Work and Recommendations}