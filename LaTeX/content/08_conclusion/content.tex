\section{Conclusion}

%This section will first summarize the findings of the study, followed by an overview about the contribution and implications of the thesis for the current state of research, and finally a recommendation for future research in the area of experimental research in data analytics is given. 
%\subsection{Discussion}

The goal of this thesis was the development of an artifact which would improve Experimental Research in Behaviroal research in Data Analytics. This goal was successfully achieved by analyzing the current state of research, establishing requirements and testing a solution, implementing an artifact, and evaluating it. Not only was an artifact developed that streamlines the execution of experiments and thus contributes to more efficient and effective research in the area of data analytics, but also literature gaps in the area of behavioral research in data analytics and in the execution of experiments were identified. Therefore, the developed artifact in the form of an Android app not only serves as a framework to better conduct experiments in the area of behaviroral research in data analytics, but the findings and requirements that were identified through it also serve as an important analysis of the state of research in the area of experimental setups.

Furthermore, these practical findings and implications also serve as a counterpart to the theoretical knowledge that can be found in various journals or books and show the challenges and particularities of how these theoretical concepts are applied in practice. To give one example, this highlights the challenge of different group assignments or the collection of participant feedback in practice. As already described in Krakowski S. et al. paper, technologies such as machine learning, \ac{ai} or data analytics already represent alternatives to human decision-making, and the artifact developed in this paper allows these \enquote{black-box} technologies to be better understood and thus used, which has a direct impact on the success of a company (\cite{Krakowski.2023}). Amankwah-Amoah and Adomako examine in their study big data usage on business failure and come to the conclusion that the correct use of data analytics and big data is an important factor in the success of a company. An improved state of research made possible by the artefact that was created within this thesis could therefore not only lead to new scientific findings but also have a direct influence on the success of companies.

The theorethical implications of research are very comprehensive, especially for the improved execution of experiments. Not only can new research be conducted more efficiently and effectively, but previous results and findings can be more easily verified and re-examined. The use of the developed artifact would make an essential contribution to the creation of new knowledge and the consolidation of already existing knowledge. The limitations of the developed artifact are mainly related to technological limitations. Due to the technology-independent development of the basic processes, the results of the work represent generally valid research. Nevertheless, the technological implementation of the developed artifact should be re-evaluated at a later point in time, should new technologies become available that might better reflect the collected and generally valid requirements.

In conclusion, in the course of this thesis it was possible not only to find a literature gap in the area of behavioral research in data analytics and the general execution of experiments, but also to identify generally applicable requirements for an application for the execution of epxeriments on the basis of various sources. In the process, new knowledge was created and existing knowledge was validated. At the same time, an artifact in the form of an Android app was developed and evaluated, which can be used to conduct experiments in the field of data analytics more efficiently and effectively. Thus, this work not only contributes to the current state of research, but also enables future researchers to better create new knowledge and consolidate existing knowledge through the developed artifact.

%\subsection{Conclusion}

%Amankwah-Amoah and Adomako 

%big data us-
%age on business failure

%(\cite{Krakowski.2023}) these black box technologies are already alternatives für decision making

%(\cite{AmankwahAmoah.2019}) untersuchen einsatz von Data analytics auf business failure


%The goal of this thesis was the development of an artifact which would improve Experimental Research in Behaviroal research in Data Analytics. For this purpose the \ac{dsr} approach was used. First, the current state of knowledge in the field of data analytics was analyized. For this a literature review was conducted, which revealed that there is a research gap in behavioral research in the area of data analytics as well as a general lack of experimental research in this area. The two problems identified, a lack of behavioral research in the field of data analytics and a lack of experiments as a method in studies, were then used to find objectives for a solution to these problems in form of an artefact. For this reason, a further literature search in the field of data analytics was used to identify literature that either already uses experiments in the experimental setup or allows their use in principle. From the analysis of this literature, in combination with other sources, requirements were finally established that the artifact must fulfill in order to be used for conducting experiments.

%\subsection{Conclusion}


%The focus was primarily on literature that deals with obstacles in the use of data analytics in the context of organizations. The results of this literature search were then assigned to the different phases of the information value chain. This assignment reveals a research gap in the field of behavioral research in the context of data analytics. 




%At the same time, a research gap in the experimental setups of the literature could be found. 

%This indicates a below average perpetuation of experiments. The two problems identified, a lack of behavioral research in the field of data analytics and a lack of experiments as a method in studies, were then used to find targets for a solution to these problems. In a new literature search, all literature in the field of data analytics was identified that either used experiments or allowed the use of experiments in principle through their methodology. In this way, literature was identified that provides information about possible problems and challenges in the use of experiments in studies in the field of data analytics. 

%Furthermore, we looked at existing solutions that can be used to conduct experiments and other sources that provided information about experiments in general. These sources were then used to establish requirements. These were first identified from the literature found and then processed using the findings from the analysis of existing applications and other sources. The requirements for the artifact have thus reflected a precise list of requirements for an application with the help of which experiments in the field of behavioral research in data analytics can be carried out. 


%Test cases for the individual requirements were also set up for the validation of the requirements. These can be used to verify the requirements themselves and at the same time serve to validate the final artifact. Subsequently, the artifact was conceptualized. For this purpose, processes for the artifact were first conceptualized in a general and technology-independent manner. These processes are based on the basic functionalities that where discovered by the requirements. Subsequently, Android Studio in combination with Java was selected as the technologies for implementing the artifact. The requirements were again the decisive factor in the selection of the technologies. In the next step, a technical architecture for the artifact was developed. Best practices as well as prerequisites defined by the requirements were taken into account. The fully conceptualized architecture was then implemented and deployed. The implementation of an example study also demostrated the functions and features of this conceptualized artifact. The implemented artifact was then validated. For this purpose, the individual test cases were checked on the one hand, and the requirements that were set up were examined in detail on the other. In addition, the user interface elements were demonstrated by means of a practical test in which 31 participants navigated through the user interface. No errors or deficiencies in the design of the user interface were found. Overall, the artifact could be validated for all requirements in this way.

%\subsection{Contributions and Implications}

%The artifact designed in this thesis represents a framework that can be used to conduct experiments in the field of behavioral research in the area of data analytics. By referencing previously conducted studies in the area of data analytics and establishing requirements based on these studies, the artifact can be used in a variety of different experimental setups in the process. The artifact provides both standard functionality that already maps the most important processes in the execution of experiments, as well as the possibility to develop your own experiment setup in the artifact itself. The experiment setup benefits directly from the already implemented infrastructure. Provided that the artifact is used to conduct experiments in future studies, it fundamentally allows a great contribution to the state of the art of research, by not only providing an experiment framework with processes based on scientific research, but also streamlining the process of developing the experiment and conducting it. In comparison to alternative application, the entiwcked artifact also provides a technology that is considered simple and extremely widespread through the use of Java in combination, which can be deployed with little effort and local. While other applications concentrate on field studies, the focus of the developed artifact lies primarily in the implementation of local laboratory experiments with participants on site. A circumstance that as discussed is preferable in the field of behavioral research. In addition to improvements in the research process of behaviroal research in the field of data analyitcs, which relys on the future usage of the artefact, the implied results from this work are important findings. Thus, both gaps in behaviroal research in the area of data analytics and in the conduct of experiments could be found. The summarized requirements of this work can serve as requirements and quality characteristics for other experimental setups independent of the developed artifact.

%\subsection{Future Work and Recommendations}

%Future research should therefore focus primarily on finally addressing the gaps in research that have been identified, using the de-iwed artifact. At the same time, it would be conceivable to extend the processes and functions of the artifact by other subareas or modules. After the productive use of the artifact in several studies, the established requirements could also be re-evaluated in order to identify potential improvements to the artifact that only occur in practical and large-scale use. In addition, it is conceivable to re-evaluate the technologies used at some point in the future. Although requirements and processes were set up in a technology-neutral way and therefore do not become obsolete, and the selected technologies were specifically analyzed for their future viability, it is very likely and inevitable that these technologies will become obsolete at some point in the future. At that point, it might therefore make sense to re-evaluate the selection of technologies and, if necessary, to re-conceptualize them in a technologically different application.