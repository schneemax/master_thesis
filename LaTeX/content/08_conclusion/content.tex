%\section{Conclusion}

%The objective of this thesis was the development of an artifact which would improve experimental research in behaviroal research in Data Analytics. This was to be ensured by the following three Objectives: (1) the review of prior research on data analytics and their methodological procedure (2) the development of an artefact which improves the research process in the field of behavioral research in data analytics (3) the validation of the artefact through the exemplary realization ofa study in said field utilizing said artefact. These three objectives were achieved.

%These objectives were successfully achieved. Not only was an artifact developed that streamlines the execution of experiments and thus contributes to more efficient and effective research in the area of data analytics, but it was also validated by implementing a sample study. Moreover, the current state of knowledge in the field of behavioral research in data analytics was analyzed, and gaps in research on behavioral and experimental research in data analytics were identified. Therefore, the developed artifact in the form of an Android application not only serves as a framework to better conduct experiments in the area of behaviroral research in data analytics, but the findings and requirements that were identified through it also serve as an important analysis of the state of research in the area of experimental setups. The three objectives that this thesis was intended to fulfill could thus be met.

%The practical findings and implications also serve as a counterpart to the theoretical knowledge that can be found in various journals or books and show the challenges and particularities of how these theoretical concepts are applied in practice. To give one example, this highlights the challenge of different group assignments or the collection of participant feedback in practice. As already described in Krakowski S. et al. (2022) paper, technologies such as machine learning, \ac{ai} or data analytics already represent alternatives to human decision-making, and the artifact developed in this thesis allows these \enquote{black-box} technologies to be better understood and thus used, which has a direct impact on the success of a company (\cite{Krakowski.2023}). Amankwah-Amoah and Adomako examine in their study big data usage on business failure and come to the conclusion that the correct use of data analytics and big data is an important factor in the success of a company. An improved state of research made possible by the artefact that was created within this thesis could therefore not only lead to new scientific findings but also have a direct influence on the success of companies.

%The theorethical implications of research are very comprehensive, especially for the improved execution of experiments. Not only can new research be conducted more efficiently and effectively, but previous results and findings can be more easily verified and re-examined. The use of the developed artifact would make an essential contribution to the creation of new knowledge and the consolidation of already existing knowledge. Accordingly, the artifact should be used in the future to conduct experiments in the field of data analytics as well as to re-evaluate already conducted studies with the help of an experimental setup. The limitations of the developed artifact are mainly related to technological limitations. Due to the technology-independent development of the basic processes, the results of the work represent generally valid research. Nevertheless, the technological implementation of the developed artifact should be re-evaluated at a later point in time, should new technologies become available that might better reflect the collected and generally valid requirements.

%In conclusion, in the course of this thesis it was possible not only to find a literature gap in the area of behavioral research in data analytics and the general execution of experiments, but also to identify generally applicable requirements for an application for the execution of epxeriments on the basis of various sources. In the process, new knowledge was created and existing knowledge was validated. At the same time, an artifact in the form of an Android application was developed and evaluated, which can be used to conduct experiments in the field of data analytics more efficiently and effectively. Thus, this work not only contributes to the current state of research, but also enables future researchers to better create new knowledge and consolidate existing knowledge through the developed artifact.

\section{Conclusion}

The objective of this thesis was the development of an artifact that would improve experimental research in behavioral research in data analytics. This was ensured by the following three objectives: (1) the review of prior research on data analytics and its methodological procedures, (2) the development of an artifact that improves the research process in the field of behavioral research in data analytics, and (3) the validation of the artifact through the exemplary realization of a study in said field utilizing said artifact. These objectives were successfully achieved. Not only was an artifact developed that streamlines the execution of experiments and thus contributes to more efficient and effective research in the area of data analytics, but it was also validated by implementing a sample study. Moreover, the current state of knowledge in the field of behavioral research in data analytics was analyzed, and gaps in research on behavioral and experimental research in data analytics were identified. Therefore, the developed artifact in the form of an Android application not only serves as a framework to better conduct experiments in the area of behavioral research in data analytics, but the findings and requirements that were identified through it also serve as an important analysis of the state of research in the area of experimental setups. The three objectives that this thesis was intended to fulfill could thus be met.

The practical findings and implications also serve as a counterpart to the theoretical knowledge that can be found in various journals or books and show the challenges and particularities of how these theoretical concepts are applied in practice. To give one example, this highlights the challenge of different group assignments or the collection of participant feedback in practice. As already described in Krakowski S. et al. (2022) paper, technologies such as \ac{ml}, \ac{ai}, or data analytics already represent alternatives to human decision-making, and the artifact developed in this thesis allows these \enquote{black-box} technologies to be better understood and thus used, which has a direct impact on the success of a company (\cite{Krakowski.2023}). Amankwah-Amoah and Adomako examine in their study big data usage on business failure and come to the conclusion that the correct use of data analytics and big data is an important factor in the success of a company. An improved state of research made possible by the artifact that was created within this thesis could therefore not only lead to new scientific findings but also have a direct influence on the success of companies.

The theoretical implications of research are very comprehensive, especially for the improved execution of experiments. Not only can new research be conducted more efficiently and effectively, but previous results and findings can be more easily verified and re-examined. The use of the developed artifact would make an essential contribution to the creation of new knowledge and the consolidation of already existing knowledge. Accordingly, the artifact should be used in the future to conduct experiments in the field of data analytics as well as to re-evaluate already conducted studies with the help of an experimental setup. The limitations of the developed artifact are mainly related to technological limitations. Due to the technology-independent development of the basic processes, the results of the work represent generally valid research. Nevertheless, the technological implementation of the developed artifact should be re-evaluated at a later point in time, should new technologies become available that might better reflect the collected and generally valid requirements.

In conclusion, in the course of this thesis, it was possible not only to find a literature gap in the area of behavioral research in data analytics and the general execution of experiments but also to identify generally applicable requirements for an application for the execution of experiments based on various sources. In the process, new knowledge was created and existing knowledge was validated. At the same time, an artifact in the form of an Android application was developed and evaluated, which can be used to conduct experiments in the field of data analytics more efficiently and effectively. Thus, this work not only contributes to the current state of research but also enables future researchers to better create new knowledge and consolidate existing knowledge through the developed artifact.
