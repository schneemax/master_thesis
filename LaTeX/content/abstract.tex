%%%%%%%%%%%%%%%%%%%%%%%%%%%%%%%%%%%%%%%%%%%%%%%%%%%%%%%%%%%%%%%%%%%%%%%%%%%%%%%%%%%%%%%%%%%
% "Abstract" Section
%%%%%%%%%%%%%%%%%%%%%%%%%%%%%%%%%%%%%%%%%%%%%%%%%%%%%%%%%%%%%%%%%%%%%%%%%%%%%%%%%%%%%%%%%%%
\sectionunnumbered{Abstract}
Over the past few years, data analytics has become increasingly important for companies across all industries. With the massive amount of data that is now available, companies can use data analytics to gain valuable insights into consumer behavior, market trends, and internal operations, among other things. As a result, data analytics has become a critical tool for companies looking to gain a competitive edge in today's rapidly evolving business environment. However, while data analytics has become an essential tool for businesses, there has been relatively little research done in the area of behavioral research. Specifically, there is a lack of research on the decision-making process involved in data analytics, and how individuals and organizations use data analytics to inform their decisions. One of the major challenges in conducting research in this area is the high cost of developing custom applications for each study. The development of such applications can be time-consuming, expensive, and often requires specialized expertise. To address this challenge, this master thesis develops a generic application that streamlines the process of conducting studies in the field of data analytics. This application enables researchers to design, conduct, and analyze studies more efficiently and cost-effectively, allowing them to explore the field in greater depth. This will be accomplished by using the design science research approach. Firstly, the problem of a lack of behavioral research in data analytics is identified. Then, the objectives for a solution are defined through a literature review and the use of requirement engineering to gather requirements for the application. Next, the application is design, implemented prototypically and its functionality demonstrated. Finally, solution is evaluated through the usages of the requirements.

\newpage