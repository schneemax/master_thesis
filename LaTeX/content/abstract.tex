%%%%%%%%%%%%%%%%%%%%%%%%%%%%%%%%%%%%%%%%%%%%%%%%%%%%%%%%%%%%%%%%%%%%%%%%%%%%%%%%%%%%%%%%%%%
% "Abstract" Section
%%%%%%%%%%%%%%%%%%%%%%%%%%%%%%%%%%%%%%%%%%%%%%%%%%%%%%%%%%%%%%%%%%%%%%%%%%%%%%%%%%%%%%%%%%%
%\sectionunnumbered{Abstract}
\section*{Abstract}
Over the past few years, data analytics has become increasingly important for companies across all industries. With the massive amount of data that is now available, companies can use data analytics to gain valuable insights into consumer behavior, market trends, and internal operations, among other things. As a result, data analytics has become a critical tool for companies looking to gain a competitive edge in today's rapidly evolving business environment. However, while data analytics has become an essential tool for businesses, there has been relatively little research done in the area of behavioral research. %Specifically, there is a lack of research on the decision-making process involved in data analytics, and how individuals and organizations use data analytics to inform their decisions. 
Specifically, there is a lack of research using experiments. At the same time, existing applications that are supposed to support the execution of experiments have many disadvantages and flaws. For this reason, the \ac{dsr} approach was used to conceptualize an artifact in the form of an Android application which streamlines the execution of experiments and thus improves the research process in the field of behavioral research in data analytics. To achieve this objective, (1) prior research on data analytics and their methodological procedure was reviewed resulting in requirements for the artifact (2) an artefact was developed which improves the research process in the field of behavioral research in data analytics using these requirements and (3) the artefact was validated through the exemplary realization of a study in data analytics utilizing said artefact. The resulting artefact can be used to conduct experiments in the field of data analytics more efficiently and effectively. Thus, this work not only contributes to the current state of research, but also enables future researchers to better create new knowledge and consolidate existing knowledge through the developed artifact. %Through this thesis, it was possible to (1) identify the current state of research in data analytics and a literature gap in behavioral and experimental research in data analytics (2) develop an artifact to improve the research process in data analytics, and (3) verify this artifact by implementing a sample study.




%One of the major challenges in conducting research in this area is the high cost of developing custom applications for each study. The development of such applications can be time-consuming, expensive, and often requires specialized expertise. To address this challenge, this master thesis develops a generic application that streamlines the process of conducting studies in the field of data analytics. This application enables researchers to design, conduct, and analyze studies more efficiently and cost-effectively, allowing them to explore the field in greater depth. This will be accomplished by using the design science research approach. Firstly, the problem of a lack of behavioral research in data analytics is identified. Then, the objectives for a solution are defined through a literature review and the use of requirement engineering to gather requirements for the application. Next, the application is design, implemented prototypically and its functionality demonstrated. Finally, solution is evaluated through the usages of the requirements.
\newpage