\newpage\section{Theoretical foundations}

This section represents the theoretical fundamentals of this elaboration by defining the term \enquote{data analytic}, the concept of the \enquote{information value chain}, the \ac{dsc} methodology, the requirement engineering approach and the experimental research process in general.

%\subsection{Definition of terms}

\subsection{Data Analytics}

The term \enquote{data analytics} originated in the early 2000s and describes an interdisciplinary field that combines areas such as statistics, machine learning, pattern recognition, system theory, operations research and artificial intelligence \parencite{Runkler.2020}. It can be generally defined \enquote{[...] as the application of computer systems to the analysis of large data sets for the support of decisions.} \parencite{Runkler.2020}. This definition showcases the broadness of the topic, as most computer systems process some amount of data and theoretically allow for some kind of decision making. Due to this broad definition, data analytics can cover slightly different subject areas depending on the context it is discussed in. In this elaboration, data analytics refers to the processing of large amounts of data, also referred  to as \enquote{big data}, through mathematical procedures or machine learning methods with the goal of creating new knowledge. Subsequently, processes that merely prepare or show data are not considered data analytics, but only processes that process data in such a way that new knowledge can be derived from it. This distinction is made to differentiate data analytics from traditional data processing areas like business intelligence. The goal of data analytics, as is discussed in this thesis, is to retrieve some kind of previously unknown knowledge from a set of data. This process can be generally described using the \enquote{information value chain} model. In their research, Abbasi et al. analyze this model in the context of big data in an effort to create an inclusive research agenda for big data in information system research \parencite{Abbasi.2016}.

\subsection{Information Value Chain}

The information value chain (figure \ref{information_value_chain}) is a set of phases that define the transformation of raw data to information and eventually into knowledge. \enquote{Data} describes raw facts without any structuring. Once organized, the processed data represents \enquote{information}. This \enquote{information} is then used to find patterns and draw conclusions. At this time, the information becomes knowledge \parencite{Fayyad.1996}, \cite{Fayyad.1996b}. This knowledge is then used to make \enquote{decisions} and take corresponding \enquote{actions} \parencite{Sharma.2014}. Each phase of the information value chain also includes a different set of technologies and methodologies. For example, the \enquote{data} phase contains technologies and actions regarding the basic storage of data like database systems or data warehouses \parencite{Abbasi.2016}. The conventional version of this information value chain represents an approach that generally explains the processing of data. The main steps of this information value chain are also applicable for big data \parencite{Abbasi.2016}. This general structure of processing data is also supported by literature from the data analytics field \parencite{Runkler.2020}. In addition, the information value chain contains the further phases \enquote{decisions} and \enquote{actions}, which deal with the influence of the processed data. These phases reflect the impact of data analytics, since data analytics is primarily a technology for the decision-making process \parencite{Runkler.2020}. For this reason, the information value chain is a suitable model to structure different phases in the processing of data in the context of data analytics. %For this reason, the literature examined in this paper is structured according to the phases of the information-value chain.



\begin{figure}[]
    \includegraphics[width=0.99\textwidth, keepaspectratio]{content/02_theretical_foundations/informationValueChain.pdf}
    \caption{Information Value Chain}    
    \label{information_value_chain}
\end{figure}

%\subsection{Boundaries and Conflicts in Organizations}

%This literature review uses the terms boundary and conflict interchangeably. In order to include as much literature as possible, the criteria for boundaries are kept very general. Prior to conducting the literature review, there was no formal definition of boundaries in the context of data analytics used for the selection of literature. Generally, boundaries are described as \enquote{[...] a real or imagined line that marks the limits or edges of something and separates it from other things or places [...]} \parencite{Hornby.2015}. Based on this general description, the term boundary is defined in the context of this elaboration as any circumstance that leads to a reduction in the effectiveness or efficiency of an organization. %Based on this description, the term \enquote{boundary} is defined in the context of this elaboration as any situation that leads to a reduction in the productivity or effectiveness of a company.
%Simultaneously, the term conflict also describes this circumstance. 
%Boundaries and conflicts are therefore used to describe any circumstance that hinders an organization from being perfectly productive. An example of such boundaries or conflicts would be communication issues between different departments, which lead to a reduction of productivity.

\subsection{Design Science Research Methodology}

\subsection{Functional and non functional Requirements}

As a means to create any artefact, the determination of requirements are important(\cite{Seacord.2003}). Requirements can be classified according to ISO/IEC 25000, respectively the quality model from ISO/IEC 25010, as quality criteria for software and systems.(\cite{ISOIEC25010.2011}). The \ac{ieee} defines requirements as:

\begin{quote}
    \textbf{\textit{\enquote{(1) A condition or capability needed by a user to solve a problem or achieve an objective. (2) A condition or capability that must be met or possessed by a system or system component to satisfy a contract, standard, specification, or other formally imposed documents. (3) A documented representation of a condition or capability as in (1) or (2). See also: design requirement; functional requirement; implementation requirement; interface requirement; performance requirement; physical equirement.}}} \cite{IEEE.1990}
\end{quote}  
   
According to these two definitions, requirements can be generally defined as properties that need to be met in order to achieve an objective. For this reason, the requirements for the artefact are derived from properties and characteristics of past studies in the field of data analytics. In addition, general requirements for an artefact which is supposed to improve the experimental research process are also taken into account. These requirements are further divided into functional and non-functional requirements. A functional requirement describes a function that a system or system component must be able to perform (\cite{IEEE.1990}). An example of a functional requirement would be the calculation of a pricetag in euros and in dollars. Non-functional characteristics describe on the other hand describe the behavior of a system (\cite{Seacord.2003}) and go thereby beyond the functional characteristics. Thus functional requirements describe what a system must be able to do and non-functional requirements describe how this should be done. Non-functional requirements also often describe the quality of the functions and can influence several other requirements (\cite{Balzert.2011}). An example of non-functional properties would be that the conversion from euros to dollars must be performed in \enquote{a few seconds}.

\subsection{Requirement Engineering}

In order to define an objective for a possible solution to the research question requirements for the created artefact must be engineered. In order to accomplishe that the \textit{requirement engineering} approach for analysis and evaluation of requirements is utilized (\cite{SWEBOK.2004}, \cite{Sommerville.2011}). This approach has been shown to clearly contribute to software project successes in the past (\cite{Hofmann.2001}) and is therefore a suitable approach to define the objectives for a solution. The exact individual phases and steps of the \textit{requirement engineering} approach can vary from source to source and use case to use case. In general, however, all steps fall into one of the three main categories. The \textit{Requirements Elicitation}, \textit{Requirements Specification} and \textit{Requirements Validation}. In the first step, possible requirements and use cases are collected with the help of analyses, surveys, literature, interviews or other sources (\cite{Sommerville.2011}). This thesis uses for the purpose of discovering requirements a literature review. In the next step, the requirements are then specified and categorized. An important distinction here is between functional and non-functional requirements. In the \textit{Requirements Validation} step, the elicited requirements are then tested for their validity. This phase emphasis the reviewing of the requirements in order to find out whether these requirements are actually representive of the desired artefact (\cite{Sommerville.2011}). This is accomplished througt Validity, Consistency, Completeness, Realism, and Verifiability checks in conjunction with protoyping and testing the requirements (\cite{Sommerville.2011}).




\subsection{Experiments and research methodologies}