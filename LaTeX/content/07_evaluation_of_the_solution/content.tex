\section{Evaluation of the Solution}

To validate the developed prototype, the previously established test cases are verified for the individual requirements. The testing of the test cases provides information about the fulfillment of the requirements. In addition, the fulfillment of requirements that could not be validated by test cases is considered. Subsequently, the usability of the \ac{ui} of the prototype is assessed.

\subsection{Prototype validation}

This section looks at the individual test cases from Section \ref{subsec:requirement_validation}. In summary, all requirements could be verified with the help of the established test cases. Table \ref{tab:FuncRequirementsCatCheck} and \ref{tab:NonFuncRequirementsCatCheck} shows the corresponding requirements and the associated test cases that prove and verify them. A detailed description of the fulfillment of the test cases is included in the appendix. One exception are the non-functional requirements N3.1 simplicity, N4.3 openness of platform and N7.1 advanced user interface, which cannot be verified by test cases due to their subjectivity. Nevertheless, as already discussed in section \ref{subsec:requirement_validation}, these requirements represent important specifications for the artifact. For this reason, the fulfillment of these non-functional requirements will be addressed as feasibly as possible without the use of test cases. For this purpose, the experience gained from the implementation of the artifact and the documentation on the individual technologies is used. Besides the fact that Android and Java are generally concidered to be simple beginner friendly technologies by the developer community, the best practice architecture that was implemented in the course of this application is a big indicator for the simplicity of the application. The individual functions and code modules have been divided into reusable UseCases and all user interface activities have been commulated into android activities. In addition, the implementation of new custom capabilities for individual experiments has been made extremely easy. For example, the definition of the experiment steps is realized through the experiment data and does not have to be implemented separately by program code. The implementation of custom experiments is also streamlined which makes it possible for the person performing the experiment to set up his experiment without having to worry about the basic framework. Despite the subjectivity of the "simplicity" requirement, the requirement N3.1 simplicity is considered to be fulfilled for the artefact due to the usage of technologies which are considered to be simple and beginner friendly, the reusable and streamlined best practice architecture and functionalities of the andoird application and the encapsulation of standard functionalities. 
Requirement N4.3 Openness of platform describes the openness of the artifact for changes and enhancements. In general, it could be shown in the already verified requirements that the artifact is extensible. By using the best practice architecture in combination with the general concept of object orientation on which Java is based, it is also possible to argue that the application can be easily enhanced.
In addition, to verify requirement N7.1, the artifact must enable the use of modern user interface componente. As already explained, this requirement is also a subjective but nevertheless important requirement for the artifact. In the user interfaces of the final artifact, the MaterialUI developed by Google is used. Google claims that the Material UI is distinguished from other user interface technologies by its Responsive Design, Motion and Animation, Consistency, Accessibility, Cross-Platform Support and other design-related features. Furthermore, MaterialUI is characterized by active development on the part of Google and is regularly provided with updates. Thus, the underlying \cite{ui} technology that was used for the development of the artifact represents a \cite{ui} technology that will deliver a modern user interface in 2023, and the regular updates by Google can be assumed to guarantee this circumstance in the long run. In summary, the N7.1 requirement is considered to be fulfilled as well as possible through the use of the MaterialUI interface in combination with its update guarantee (\cite{Google.2023c}, \cite{Google.2023}). In conclusion, all requirements could be verified using and fulfilling the corresponding test cases. In addition, it was possible to show argumentatively on the basis of various sources why all other requirements that cannot be substantiated by test cases are also considered to be fulfilled and verified. Hence, all requirements for the artifact are fulfilled and verified.

\begin{table}
    \centering
    \small
    \begin{tabular}{L{0.35\textwidth}C{0.19\textwidth}C{0.19\textwidth}}
    \hline
    Requirement                     & Testcase & Fulfilled \\ \hline
    \rowcolor[HTML]{C0C0C0} 
    Information                     &   &          \\ \hline
    F1.1 Displaying Information     &  T1 &  \faCheck         \\
    % & \\
    F1.2 Debrefing Info             & T1 &  \faCheck      \\\hline
    \rowcolor[HTML]{C0C0C0} 
    Data Collecting             &   &            \\ \hline
    F2.1 Participant Data           & T2, T5 &   \faCheck       \\
    F2.2 Meta-Data                  &  T3, T8 &  \faCheck        \\
    F2.3 Post-Interview             & T2, T5 &   \faCheck       \\\hline
    \rowcolor[HTML]{C0C0C0} 
    Pre-Loading                 &  &            \\ \hline
    F3.1 Pre-Loading Data           & T4 & \faCheck        \\
    F3.2 Selecting Data             & T4 &  \faCheck       \\\hline
    \rowcolor[HTML]{C0C0C0} 
    Experiment Setup            &    &         \\ \hline
    F4.1 Additional Logic           & T5 & \faCheck       \\
    F4.2 Participant Input          & T2, T5, T7 &  \faCheck         \\
    F4.3 Proactive System           & T2, T5 &   \faCheck      \\\hline
    \rowcolor[HTML]{C0C0C0} 
    Groups                      & &            \\ \hline
    F5.1 Different Groups           & T6 &  \faCheck    \\
    F5.2 Communication of Groups    & T7  & \faCheck     \\ 
    F5.3 Targeted Assignment        & T6 & \faCheck \\
    F5.4 Random Assignment          & T6 & \faCheck \\ \hline
    \end{tabular}
    \caption[Fulfillment of Functional Requirements]{Fulfillment of Functional Requirements}\label{tab:FuncRequirementsCatCheck}
    \end{table}

\begin{table}
    \centering
    \small
    \begin{tabular}{L{0.35\textwidth}C{0.19\textwidth}C{0.19\textwidth}}
    \hline
Requirement                             & Testcase & Fulfilled  \\ \hline
    \rowcolor[HTML]{C0C0C0} 
    Time-space non-reliance     &   &          \\ \hline
    N1.1 Distand Communication      & T7 & \faCheck           \\
    N1.2 Time-Flexibility           & T7 & \faCheck            \\ \hline
    \rowcolor[HTML]{C0C0C0} 
    Data Postprocessing &      &       \\ \hline
    N2.1 Evaluation of Data         &  T8 &   \faCheck          \\
    N2.2 Vizualize Final Data       & T8 &  \faCheck         \\ \hline
    \rowcolor[HTML]{C0C0C0} 
    Simplicity                  &    &        \\ \hline
    N3.1 Simplicity                 &  &        \\\hline
    \rowcolor[HTML]{C0C0C0} 
    Reusable and Interoperable  &       &      \\ \hline
    N4.1 Reusable                   & T9 &  \faCheck         \\
    N4.2 Interoperability           & T10 &  \faCheck        \\
    N4.3 Openness of Platform       & &  \\\hline
    \rowcolor[HTML]{C0C0C0} 
    Monitoring                  &  &           \\ \hline
    N5.1 Monitoring of Study        & T8, T11 &  \faCheck         \\ \hline
    \rowcolor[HTML]{C0C0C0} 
    Pre-Loading                 &     &        \\ \hline
    N6.1 Multi-Source             & T4 &  \faCheck          \\ \hline
    \rowcolor[HTML]{C0C0C0} 
    Advanced User Interface   &    &         \\ \hline
    N7.1 Advanced User Interface  & &          \\ \hline
    \end{tabular}
    \caption[Fulfillment of Non-Functional Requirements]{Fulfillment of Non-Functional Requirements}\label{tab:NonFuncRequirementsCatCheck}
    \end{table}



\subsection{App Performance and Usability}

After all the requirements for the artifact have been met and the functionality of the artifact has been demonstrated, this section briefly validates the UI of the application. As described in this thesis, the UI of the artifact was implemented based on the different processes that require user inputs. For this purpose, the respective activities in the said processes were used and implemented on the individual screens by the corresponding UI components of an Android application with Material UI design. The UI thus adheres to the proven and tested \ac{ui} concepts of Google's Material \ac{ui} design philosophy. Nevertheless, this part of the work is intended to test the rough layout and usability of the artifact's user interface. The goal of this section is not to identify the best possible or most beautiful \ac{ui}, but to verify that the \ac{ui} used is a usable interface. For this purpose, a paper prototype is built using the screenshots from the artifact with the help of SAP Build. This prototype will be sent to 50 business informatics students of a large German Dax software company. They have the task to click through the prototype. Both the clicks and the time needed to navigate through the individual screens and the prototype are measured. The goal of this small test is to verify that the UI of the prototype is basically usable. The test would show potential for improvement if some participants did not manage to click to the end of the paper prototype, took an unusually long time to do so, or clicked several times at a point that was not intended to be interact with. In total, 32 of the invited participants were able to take part in the study. The participants took part in the test anonymously, for this reason the names of the participants in Appendix \ref{appendix:heatmapTimes} are filled with placeholders. The participants spent an average of 23.9375 seconds to complete the paper prototype. The users' summarized clicks are shown in the heatmaps in Figure \ref{subfig:heatmapA}, \ref{subfig:heatmapB}, and \ref{subfig:heatmapC} in the appendix. The results of the test do not indicate any serious negative design decisions. The average time it took the participants to click through the prototype is reasonable and the heatmap in Figure \ref{fig:heatmaps} does not show any unusual hotspots. In general, the user interface appears to be usable and did not present any major or new challenges to the participants of the test. Once more, it should be pointed out that this test only serves to verify that the activities from the processes in connection with Google Material UI result in a meaningful and usable user interface and that the usability properties claimed by Google about MaterialUI can be fulfilled in the context of the artifact. Nevertheless, the test shows that the user interface of the artifact is usable and, in combination with the claims made by Google about the MaterialUI, is considered sufficiently functional for the artifact. 

%In summary, this section verified the developed artifact. The requirements placed on the artifact were checked with the help of test cases and other sources. All requirements for the artifact were met. In addition, the \ac{ui} designed from the identified process steps and Google's MaterialUI was tested using a user test. This showed that the artifact has a usable interface and that the individual Android activities have no obvious potential for improvement. 