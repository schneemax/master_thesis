%\section{Evaluation of the Solution}

%To validate the developed prototype, the previously established test cases are verified for the individual requirements. In addition, the fulfillment of requirements N3.1 (Simplicity), N4.3 (Openness of Platform) and N7.1 (Advanced User Interface) that could not be validated by test cases is discussed. Subsequently, the usability of the \ac{ui} of the prototype is assessed via a user interface test.

%\subsection{Prototype validation}

%This section looks at the individual test cases from Section \ref{subsec:requirement_validation}. In summary, all requirements could be verified with the help of the established test cases. Table \ref{tab:FuncRequirementsCatCheck} and \ref{tab:NonFuncRequirementsCatCheck} show the corresponding requirements and the associated test cases that prove and verify them. A detailed description of the fulfillment of each of the test cases is included in appendix \ref{appendix:D}. One exception to this are the non-functional requirements N3.1 (Simplicity), N4.3 (Openness of Platform) and N7.1 (Advanced User Interface), which cannot be verified by test cases alone due to their subjectivity. Nevertheless, as already discussed in section \ref{subsec:requirement_validation}, these requirements represent important specifications for the artifact. For this reason, the fulfillment of these non-functional requirements will be addressed as feasibly as possible without the use of test cases. For this purpose, the experience gained from the implementation of the artifact and the documentation on the technologies utilitied is used. Besides the fact that Android and Java are generally concidered to be simple beginner friendly technologies (\cite{Ullenboom.2017}), the best practice architecture that was implemented in the course of this application is a big indicator for the simplicity of the application (\cite{Google.2023}). The individual functions and code modules have been divided into reusable use case and all user interface activities have been commulated into android activities. In addition, the implementation of new custom capabilities for individual experiments has been made extremely easy by the general architecture that was conceptualized. For example, the definition of a new order of \enquote{experiment steps} is realized within the corresponding data of the experiment and does not have to be implemented separately by program code. The implementation of custom experiments is also streamlined which makes it possible for the person performing the experiment to set up his experiment without having to worry about the basic tasks, like group allocation or data completion. Due to these conditions the requirement N3.1 (Simplicity) is considered to be fulfilled by the artefact.

%The last non-functional requirement which could not be validated by a test case alone is the requirement N7.1 (Advanced User Interface). The conceptualized android application utilities MaterialUI developed by Google for the \ac{ui} elements. Google claims that MaterialUI is distinguished from other user interface technologies by its Responsive Design, Motion and Animation, Consistency, Accessibility, Cross-Platform Support and other design-related features. Furthermore, MaterialUI is characterized by active development on the part of Google and is regularly provided with updates. Thus, the underlying \ac{ui} technology that was used for the development of the artifact represents a \ac{ui} technology that will deliver a modern user interface as of 2023, and the regular updates by Google can be assumed to guarantee this circumstance in the long run. In summary, the N7.1 requirement is considered to be fulfilled as well as possible through the use of the MaterialUI interface in combination with its update guarantee by Google (\cite{Google.2023c}, \cite{Google.2023}). 

%In conclusion, all requirements could be verified using and fulfilling the corresponding test cases. In addition, it was possible to show argumentatively on the basis of various sources that the non-functional requirements  N3.1 (Simplicity), N4.3 (Openness of Platform) and N7.1 (Advanced User Interface) are also fulfilled and verified. Hence, all requirements for the artifact are fulfilled and verified.

\section{Evaluation of the Solution}

To validate the developed prototype, the previously established test cases are verified for the individual requirements. In addition, the fulfillment of requirements N3.1 (Simplicity), N4.3 (Openness of Platform), and N7.1 (Advanced User Interface), which could not be validated by test cases, is discussed. Subsequently, the usability of the \ac{ui} of the prototype is assessed via a user interface test.

\subsection{Prototype Validation}

This section looks at the individual test cases from Section \ref{subsec:requirement_validation}. In summary, all requirements could be verified with the help of the established test cases. Tables \ref{tab:FuncRequirementsCatCheck} and \ref{tab:NonFuncRequirementsCatCheck} show the corresponding requirements and the associated test cases that prove and verify them. A detailed description of the fulfillment of each of the test cases is included in Appendix \ref{appendix:D}. One exception to this is the non-functional requirements N3.1 (Simplicity), N4.3 (Openness of Platform), and N7.1 (Advanced User Interface), which cannot be verified by test cases alone due to their subjectivity. Nevertheless, as already discussed in Section \ref{subsec:requirement_validation}, these requirements represent important specifications for the artifact. For this reason, the fulfillment of these non-functional requirements will be addressed as feasibly as possible without the use of test cases. For this purpose, the experience gained from the implementation of the artifact and the documentation on the technologies utilized are used. Besides the fact that Android and Java are generally considered to be simple, beginner-friendly technologies (\cite{Ullenboom.2017}), the best practice architecture that was implemented in the course of this application is a big indicator of the simplicity of the application (\cite{Google.2023}). The individual functions and code modules have been divided into reusable use cases, and all user interface activities have been consolidated into Android activities. In addition, the implementation of new custom capabilities for individual experiments has been made extremely easy by the general architecture that was conceptualized. For example, the definition of a new order of "experiment steps" is realized within the corresponding data of the experiment and does not have to be implemented separately by program code. The implementation of custom experiments is also streamlined, which makes it possible for the person performing the experiment to set up their experiment without having to worry about basic tasks like group allocation or data completion. Due to these conditions, the requirement N3.1 (Simplicity) is considered to be fulfilled by the artifact.

The last non-functional requirement which could not be validated by a test case alone is the requirement N7.1 (Advanced User Interface). The conceptualized Android application utilizes MaterialUI developed by Google for the \ac{ui} elements. Google claims that MaterialUI is distinguished from other user interface technologies by its Responsive Design, Motion and Animation, Consistency, Accessibility, Cross-Platform Support, and other design-related features. Furthermore, MaterialUI is characterized by active development on the part of Google and is regularly provided with updates. Thus, the underlying \ac{ui} technology that was used for the development of the artifact represents a \ac{ui} technology that will deliver a modern user interface as of 2023, and the regular updates by Google can be assumed to guarantee this circumstance in the long run. In summary, the N7.1 requirement is considered to be fulfilled as well as possible through the use of the MaterialUI interface in combination with its update guarantee by Google (\cite{Google.2023c}, \cite{Google.2023}). 

In conclusion, all requirements could be verified using and fulfilling the corresponding test cases. In addition, it was possible to show argumentatively on the basis of various sources that the non-functional requirements N3.1 (Simplicity), N4.3 (Openness of Platform), and N7.1 (Advanced User Interface) are also fulfilled and verified. Hence, all requirements for the artifact are fulfilled and verified.


\begin{table}
    \centering
    \small
    \begin{tabular}{L{0.35\textwidth}C{0.19\textwidth}C{0.19\textwidth}}
    \hline
    Requirement                     & Testcase & Fulfilled \\ \hline
    \rowcolor[HTML]{C0C0C0} 
    Information                     &   &          \\ \hline
    F1.1 Displaying Information     &  T1 &  \faCheck         \\
    % & \\
    F1.2 Debrefing Info             & T1 &  \faCheck      \\\hline
    \rowcolor[HTML]{C0C0C0} 
    Data Collecting             &   &            \\ \hline
    F2.1 Participant Data           & T2, T5 &   \faCheck       \\
    F2.2 Meta-Data                  &  T3, T8 &  \faCheck        \\
    F2.3 Post-Interview             & T2, T5 &   \faCheck       \\\hline
    \rowcolor[HTML]{C0C0C0} 
    Pre-Loading                 &  &            \\ \hline
    F3.1 Pre-Loading Data           & T4 & \faCheck        \\
    F3.2 Selecting Data             & T4 &  \faCheck       \\\hline
    \rowcolor[HTML]{C0C0C0} 
    Experiment Setup            &    &         \\ \hline
    F4.1 Additional Logic           & T5 & \faCheck       \\
    F4.2 Participant Input          & T2, T5, T7 &  \faCheck         \\
    F4.3 Proactive System           & T2, T5 &   \faCheck      \\\hline
    \rowcolor[HTML]{C0C0C0} 
    Groups                      & &            \\ \hline
    F5.1 Different Groups           & T6 &  \faCheck    \\
    F5.2 Communication of Groups    & T7  & \faCheck     \\ 
    F5.3 Targeted Assignment        & T6 & \faCheck \\
    F5.4 Random Assignment          & T6 & \faCheck \\ \hline
    \end{tabular}
    \caption[Fulfillment of Functional Requirements]{Fulfillment of Functional Requirements}\label{tab:FuncRequirementsCatCheck}
    \end{table}

\begin{table}
    \centering
    \small
    \begin{tabular}{L{0.35\textwidth}C{0.19\textwidth}C{0.19\textwidth}}
    \hline
Requirement                             & Testcase & Fulfilled  \\ \hline
    \rowcolor[HTML]{C0C0C0} 
    Time-space non-reliance     &   &          \\ \hline
    N1.1 Distand Communication      & T7 & \faCheck           \\
    N1.2 Time-Flexibility           & T7 & \faCheck            \\ \hline
    \rowcolor[HTML]{C0C0C0} 
    Data Postprocessing &      &       \\ \hline
    N2.1 Evaluation of Data         &  T8 &   \faCheck          \\
    N2.2 Vizualize Final Data       & T8 &  \faCheck         \\ \hline
    \rowcolor[HTML]{C0C0C0} 
    Simplicity                  &    &        \\ \hline
    N3.1 Simplicity                 &  &        \\\hline
    \rowcolor[HTML]{C0C0C0} 
    Reusable and Interoperable  &       &      \\ \hline
    N4.1 Reusable                   & T9 &  \faCheck         \\
    N4.2 Interoperability           & T10 &  \faCheck        \\
    N4.3 Openness of Platform       & &  \\\hline
    \rowcolor[HTML]{C0C0C0} 
    Monitoring                  &  &           \\ \hline
    N5.1 Monitoring of Study        & T8, T11 &  \faCheck         \\ \hline
    \rowcolor[HTML]{C0C0C0} 
    Pre-Loading                 &     &        \\ \hline
    N6.1 Multi-Source             & T4 &  \faCheck          \\ \hline
    \rowcolor[HTML]{C0C0C0} 
    Advanced User Interface   &    &         \\ \hline
    N7.1 Advanced User Interface  & &          \\ \hline
    \end{tabular}
    \caption[Fulfillment of Non-Functional Requirements]{Fulfillment of Non-Functional Requirements}\label{tab:NonFuncRequirementsCatCheck}
    \end{table}



%\subsection{Application Performance and Usability}

%After all the requirements for the artifact have been fulfilled and the functionality of the artifact has been demonstrated, this section briefly validates the \ac{ui} of the application. As described in this thesis, the \ac{ui} of the artifact was implemented based on the different \enquote{experiment process steps} that require user inputs. For this purpose, the respective activities in the processes were used and implemented on the individual screens by the corresponding \ac{ui} components of an Android application using the MaterialUI design. The \ac{ui} thus adheres to the proven and tested \ac{ui} concepts of Google's MaterialUI design philosophy. Nevertheless, this part of the thesis is intended to test the rough layout and usability of the artifact's user interface. The goal of this section is not to identify the best possible or most beautiful \ac{ui}, but to verify that the \ac{ui} used is a suitable interface for the artefact and its processes. For this purpose, a clickable mock-up prototype is built using screenshots of the artifact. The prototype feature of SAP Build.me was used for this purpose (\cite{SAP.2023}). This prototype was then clicked through by business informatics students working at a large German Dax software company. Their task simply being to click through the prototype. Both the clicks and the time needed to navigate through the individual screens and the prototype were measured. The goal of this test is to verify that the \ac{ui} of the prototype is fundamentally usable and to show any flaws or disadvantages of it. The test would also show potential for improvement if some participants did not manage to click to the end of the prototype, took an unusually long time to do so, or clicked several times at a point that was not intended to be interact with. In total, 25 individual participants took part in the study. The participants took part in the test anonymously. The results of this test do not indicate any negative design decisions. The average time it took the participants to click through the prototype is reasonable and the recorded clicks do not show any unusual hotspots or anomalies. A full list of the anonymized participants and an overview of the clicks they did is included in appendix \ref{appendix:D}. In general, the user interface appears to be usable and did not present any major or new challenges to the participants of the test. Once more, it should be pointed out that this test only serves to verify that the activities from the processes in connection with Google MaterialUI results in a meaningful and usable user interface and that the usability properties claimed by Google about MaterialUI can be fulfilled in the context of the artifact. Nevertheless, the test shows that the user interface of the artifact is usable and, in combination with the claims made by Google about the MaterialUI, is considered sufficiently functional for the artifact. 

\subsection{Application Performance and Usability}

After all the requirements for the artifact have been fulfilled and the functionality of the artifact has been demonstrated, this section briefly validates the \ac{ui} of the application. As described in this thesis, the \ac{ui} of the artifact was implemented based on the different \enquote{experiment process steps} that require user inputs. For this purpose, the respective activities in the processes were used and implemented on the individual screens by the corresponding \ac{ui} components of an Android application using the MaterialUI design. The \ac{ui} thus adheres to the proven and tested \ac{ui} concepts of Google's MaterialUI design philosophy. Nevertheless, this part of the thesis is intended to test the rough layout and usability of the artifact's user interface. The goal of this section is not to identify the best possible or most beautiful \ac{ui}, but to verify that the \ac{ui} used is a suitable interface for the artifact and its processes. For this purpose, a clickable mock-up prototype is built using screenshots of the artifact. The prototype feature of SAP Build.me was used for this purpose (\cite{SAP.2023}). This prototype was then clicked through by business informatics students working at a large German DAX software company. Their task simply being to click through the prototype. Both the clicks and the time needed to navigate through the individual screens and the prototype were measured. The goal of this test is to verify that the \ac{ui} of the prototype is fundamentally usable and to show any flaws or disadvantages of it. The test would also show potential for improvement if some participants did not manage to click to the end of the prototype, took an unusually long time to do so, or clicked several times at a point that was not intended to be interacted with. In total, 25 individual participants took part in the study. The participants took part in the test anonymously. The results of this test do not indicate any negative design decisions. The average time it took the participants to click through the prototype is reasonable, and the recorded clicks do not show any unusual hotspots or anomalies. A full list of the anonymized participants and an overview of the clicks they did is included in appendix \ref{appendix:D}. In general, the user interface appears to be usable and did not present any major or new challenges to the participants of the test. Once more, it should be pointed out that this test only serves to verify that the activities from the processes in connection with Google MaterialUI result in a meaningful and usable user interface and that the usability properties claimed by Google about MaterialUI can be fulfilled in the context of the artifact. Nevertheless, the test shows that the user interface of the artifact is usable and, in combination with the claims made by Google about MaterialUI, is considered sufficiently functional for the artifact.
