\newpage\section{Evaluation of the solution}

To validate the developed prototype, the previously established test cases are verified for the individual requirements. The testing of the test cases provides information about the degree of fulfillment of the requirements. In addition, the fulfillment of requirements that could not be validated by test cases is considered. Subsequently, the usability of the UI of the prototype will be assessed.

\subsection{Prototype validation}
This section looks at the individual test cases from Section \ref{subsec:requirement_validation} and describes both their implementation and the degree to which they have been fulfilled. Furthermore, the fulfillment of requirements that cannot be confirmed by test cases is evaluated. 

\subsubsection*{T1: A welcome and goodbye message is displayed} 

The implementation of this test case can be implemented without further development by the implemented framework. For this purpose, the Information Screen Activity is specified as the first and last step in the experiment data. Furthermore, a corresponding welcome and farewell text is stored in the experiment data. Appendix \ref{appendix:testcases} shows screenshots of the complete fulfillment of this test case.


\begin{figure}[htbp]
    \centering
    \begin{subfigure}[b]{0.3\textwidth}
        \centering
        \includegraphics[width=\textwidth]{content/07_evaluation_of_the_solution/Screenshot_StartingScreen.png}
        \caption{Welcome Message}
        \label{subfig:welcomeMessage}
    \end{subfigure}
    \hspace{1cm}
    \begin{subfigure}[b]{0.3\textwidth}
        \centering
        \includegraphics[width=\textwidth]{content/07_evaluation_of_the_solution/Screenshot_GoodbyeMessage.png}
        \caption{Goodbye Message}
        \label{subfig:goodbyeMessage}
    \end{subfigure}
    \caption{User Interface Testcase T1}
    \label{fig:T1}
\end{figure}

\subsubsection*{T2: Participants are prompted to input their age at the beginning and prompted to input how the liked the experiment at the end}


\begin{figure}[htbp]
    \centering
    \begin{subfigure}[b]{0.3\textwidth}
        \centering
        \includegraphics[width=\textwidth]{content/07_evaluation_of_the_solution/Screenshot_T2a.png}
        \caption{Welcome Message}
        \label{subfig:welcomeMessage}
    \end{subfigure}
    \hspace{1cm}
    \begin{subfigure}[b]{0.3\textwidth}
        \centering
        \includegraphics[width=\textwidth]{content/07_evaluation_of_the_solution/Screenshot_T2b.png}
        \caption{Goodbye Message}
        \label{subfig:goodbyeMessage}
    \end{subfigure}
    \caption{User Interface Testcase T1}
    \label{fig:T1}
\end{figure}

\subsubsection*{T3: The information about how long the experiment took is collected}

\begin{lstlisting}[language=java,label=t3a,lineskip={0pt}, caption=Collect time needed to conduct experiment (a), basicstyle=\scriptsize, captionpos=b]
    String currentTime = new SimpleDateFormat("HH:mm:ss", Locale.getDefault()).format(new Date());
    LogMetaDataUseCase.getInstance().setMetaData(currentTime);
\end{lstlisting}

\begin{lstlisting}[language=java,label=t3b,lineskip={0pt}, caption=Collect time needed to conduct experiment (b), basicstyle=\scriptsize, captionpos=b]
    long difference = date1.getTime() - date2.getTime();
    LogMetaDataUseCase.getInstance().setMetaData(difference);
\end{lstlisting}


\subsubsection*{T4: The gender and the weight of the participant is pre-loaded into the experiment from different files. The gender of the participant is deleted}

\begin{lstlisting}[language=java,label=t3b,lineskip={0pt}, caption=Collect time needed to conduct experiment (b), basicstyle=\scriptsize, captionpos=b]
    File csvfile = new File(Environment.getExternalStorageDirectory() + "/participantData.csv");
    CSVReader reader = new CSVReader(new FileReader(csvfile.getAbsolutePath()));
    String[] nextLine;
    while ((nextLine = reader.readNext()) != null) {
        // nextLine[] is an array of values from the line
        ParticipantEntity participant = new ParticipantEntity(Integer.parseInt(nextLine[0]));
        participant.setName(nextLine[0]);
        participant.setEducation(nextLine[0]);
        participant.setGender(null);
    
        data.add(participant);
    }
\end{lstlisting}

\subsubsection*{T5: A chess game is added as custom logic}

\begin{figure}[htbp]
    \centering
    \includegraphics[width=0.3\textwidth, keepaspectratio]{content/07_evaluation_of_the_solution/Screenshot_Chess.png}
    \caption{Custom experiment logic}    
    \label{fig:chess}
\end{figure}

\subsubsection*{T6: Two groups are created, one of the groups is particularly chosen the other one randomly selected}

%the first one through putting the group into the respective participant data
%second through code within the use case:

\begin{lstlisting}[language=java,label=t3b,lineskip={0pt}, caption=Collect time needed to conduct experiment (b), basicstyle=\scriptsize, captionpos=b]
    currentParticipant = getCurrentParticipantUseCase.getCurrentParticipant();
    currentParticipantGroup = participantRepository.getParticipant(currentParticipant).getGroupAllocation();
    
    if(currentParticipantGroup == null){
    
        if(experimentRepository.getExperiment().getGroupAllocation() == "random"){
            Random random = new Random();
            int randomNumber = random.nextInt(2); // Generates either 0 or 1
    
            if(randomNumber == 0 ){
                setGroupAllocationPUseCase.setGroupAllocation("Group A", currentParticipant);
            } else {
                setGroupAllocationPUseCase.setGroupAllocation("Group B", currentParticipant);
            }
    
        };
    
    }
\end{lstlisting}

\subsubsection*{T7: A chess turn is played by both parties not using the same device}

\subsubsection*{T8 :The results of the experiment are retrieved and displayed in third party software}


\begin{lstlisting}[language=java,label=t3b,lineskip={0pt}, caption=Collect time needed to conduct experiment (b), basicstyle=\scriptsize, captionpos=b]
    File file = new File(Environment.getExternalStorageDirectory() + "/participant" + GetCurrentParticipantUseCase.getInstance().getCurrentParticipant() + "TimeData.csv");
    try {
        // create FileWriter object with file as parameter
        FileWriter outputfile = new FileWriter(file);

        // create CSVWriter object filewriter object as parameter
        CSVWriter writer = new CSVWriter(outputfile);

        // adding header to csv
        String[] header = { "id", "time in milliseconds"};
        writer.writeNext(header);

        //Getting participant information
        ArrayList<ParticipantEntity> participantEntities = GetParticipantDataUseCase.getInstance().getParticipantData();

        //Writing current participant time into file
        //String[] data = {String.valueOf(GetCurrentParticipantUseCase.getInstance().getCurrentParticipant()), String.valueOf(timeDifference)};


        Iterator iter = participantEntities.iterator();
        while (iter.hasNext()) {
            String[] data = {String.valueOf(((ParticipantEntity)iter.next()).getId()), String.valueOf(((ParticipantEntity)iter.next()).getExperimentTime())};
            writer.writeNext(data);
        }
        //closing writer connection
        writer.close();
    }
    catch (IOException e) {
        // TODO Auto-generated catch block
        e.printStackTrace();
        System.out.println("Error");
    }
\end{lstlisting}

\begin{figure}[htbp]
    \centering
    \includegraphics[width=0.3\textwidth, keepaspectratio]{content/07_evaluation_of_the_solution/ExcelPicture.png}
    \caption{Data loaded into excel}    
    \label{fig:Excel}
\end{figure}


\subsubsection*{T9: The experiment is redone a second time and another experimental setup is implemented}

%just change the experiment data

    %\begin{lstlisting}[language=java,label=t3b,lineskip={0pt}, caption=Collect time needed to conduct experiment (b), basicstyle=\scriptsize, captionpos=b]
    %ArrayList<String> steps = new ArrayList<String>();

    %steps.add("com.example.master_thesis.InfoScreenActivity");
    %steps.add("com.example.master_thesis.ChooseTestSubjectActivity");
    %steps.add("com.example.master_thesis.QuestionnaireActivity");
    %steps.add("com.example.master_thesis.ChessExperimentActivity");
%\end{lstlisting}

\subsubsection*{T10: The experiment is conducted on different devices}

%PIXEL 6 PRO API 30

\begin{figure}[htbp]
    \centering
    \begin{subfigure}[b]{0.3\textwidth}
        \centering
        \includegraphics[width=\textwidth]{content/07_evaluation_of_the_solution/Screenshot_T10a.png}
        \caption{Info screen step | Pixel 6 Pro}
        \label{subfig:InfoScreenPixel}
    \end{subfigure}
    \hfill
    \begin{subfigure}[b]{0.3\textwidth}
        \centering
        \includegraphics[width=\textwidth]{content/07_evaluation_of_the_solution/Screenshot_T10b.png}
        \caption{ Questionair step | Pixel 6 Pro}
        \label{subfig:QuestionairPixel}
    \end{subfigure}
    \hfill
    \begin{subfigure}[b]{0.3\textwidth}
        \centering
        \includegraphics[width=\textwidth]{content/07_evaluation_of_the_solution/Screenshot_T10c.png}
        \caption{Choose test subject step | Pixel 6 Pro}
        \label{subfig:chooseTestSubjectPixel}
    \end{subfigure}
       \caption{Artefact run on a Pixel 6 Pro}
       \label{fig:uiScreens}
\end{figure}

\subsubsection*{T11: During the experiment the current state of the chess board is exported to the conducter of the experiment}

\subsubsection*{Remaining requirements}



%\subsection{Prototype Testing}



\subsection{App Performance and Usability}
